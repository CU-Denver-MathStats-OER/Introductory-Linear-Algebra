\documentclass[xcolor=dvipsnames,aspectratio=169,t]{beamer}
  % t means frames are vertically centered to the top
\usepackage{slides-header}
\title{The Inverse of a Matrix}

\begin{document}
\maketitle

\begin{frame}{The Inverse of an Operation}

  Consider solving the following linear equation with one variable:
  \begin{align*}
    2x &= 7\\
    \onslide<2->{\colorr{\frac{1}{2}} 2x &= \colorr{\frac{1}{2}} 7 \\}
     x &= \frac{7}{2}
  \end{align*}
  \smallskip
  
  \onslide<3->{
  Multiplying by $\colorr{\frac{1}{2}}$ is the \alert{inverse} of the operation of multiplying by $2$.
  \bigskip
  }
  
  \onslide<4->{
  We can solve $ax=b$ ($a,b$ real) whenever $a$ has a \colorb{multiplicative inverse}, %
  }%
  \onslide<5->{%
  which is when $a\ne 0$.
  }
  \vspace*{2em}
  
  \onslide<6->{
  \qquad Can we solve the \alert{matrix equation} $A\x=\b$ in a similar way?
  }

\end{frame}

\begin{frame}{The Inverse of a Matrix}

\bbox
Let $A$ denote a \alert{square} $n \times n$ matrix. The \alert{inverse} of $A$ (if it exists) is denoted  $A^{-1}$ and it is the unique matrix such that
\bi
\ii $\dsty AA^{-1} = I_n$ and
\ii $\dsty A^{-1}A = I_n$.
\ei
\ebox

For example if $\dsty A = \begin{bmatrix} 1 & 0 \\ 0 & 5 \end{bmatrix}$, then
$A^{-1} = \begin{bmatrix} 1 & 0 \\ 0 & \frac{1}{5} \end{bmatrix}$.
\bigskip

Check:
\[ A A^{-1} = \begin{bmatrix} 1 & 0 \\ 0 & 5 \end{bmatrix} \begin{bmatrix} 1 & 0 \\ 0 & \frac{1}{5} \end{bmatrix} = \phantom{MMMMMMMMMMMMMMMMMMM}\]
\end{frame}

\begin{frame}{Solving the Matrix Equation $A\mathbf{x}=\mathbf{b}$}
  \begin{theorem}
    If $A$ is an invertible $n\times n$ matrix, then for each $\mathbf{b}$ in $\mathbb{R}^n$,
    the matrix equation $A\mathbf{x}=\mathbf{b}$ has the \alert{unique solution} $\mathbf{x}=A^{-1}\mathbf{b}$.
  \end{theorem}
  
  \begin{proof}
  \pause
    First, we show that $A^{-1}\mathbf{b}$ is a solution:
    \[
      A(A^{-1}\mathbf{b})=(AA^{-1})\mathbf{b}=I_n \mathbf{b} = \mathbf{b}.
    \]
    
  \pause
    Second, we show the solution is unique.  Suppose $\mathbf{y}$ is a solution.
    Then $A\mathbf{y}=\mathbf{b}$.
    Multiplying both sides on the left by $A^{-1}$ and simplifying, we obtain:
    \[
      A^{-1}(A\mathbf{y})=A^{-1}\mathbf{b}
      \quad\alert{\Rightarrow}\quad
      (A^{-1}A)\mathbf{y}=A^{-1}\mathbf{b}
      \quad\alert{\Rightarrow}\quad
      I_n \mathbf{y}=A^{-1}\mathbf{b}
      \quad\alert{\Rightarrow}\quad
      \mathbf{y}=A^{-1}\mathbf{b}.
    \]
  \end{proof}

  
\end{frame}


\begin{frame}{Properties of Inverses}
  \bbox
  If $A$ is an invertible matrix, then $A^{-1}$ is invertible and its inverse is 
  \[ \left( A^{-1} \right)^{-1} = A .\]
  \ebox
  
  Why?

  \vspace{5em}
  
  \bbox
  \bi 
  \ii A matrix that is \alert{not invertible} is called a \alert{singular matrix}.
  \ii A matrix that \alert{is invertible} is called a \alert{nonsingular matrix}.
  \ei
  \ebox
\end{frame}

\begin{frame}{Properties of Inverses}
  \bbox
  If $A$ and $B$ are $n \times n$ invertible matrices, then $AB$ is an $n \times n$ invertible matrix. The inverse $(AB)^{-1}$ is obtained by taking the \alert{product of inverses in the reverse order}:
  \[ (AB)^{-1} = B^{-1} A^{-1}.\]
  \ebox
  Why?

  \vspace{5em}

%   \bbox
%   The product of $n \times n$ invertible matrices is invertible, and the inverse is the product of their inverses in the reverse order.
%   \ebox
\end{frame}

\begin{frame}{Properties of Inverses}
  \bbox
    If $A$ is an invertible matrix, then so is $A^T$, and the inverse is 
    \[ \left( A^T \right)^{-1} = \left( A^{-1} \right)^T .\]
  \ebox
  Why?
\end{frame}

\begin{frame}{Elementary Matrices}
  \bbox
    An \alert{elementary matrix} is one that is obtained by performing an elementary row operation on an identity matrix.
  \ebox

  \[ 
    E_1 =  \begin{bmatrix} 1 & 0 \\ 0 & 5 \end{bmatrix}  \qquad
    E_2 =  \begin{bmatrix} 1 & 0  & 0 \\ 0 & 1 & -2 \\ 0 & 0 & 1 \end{bmatrix}  \qquad
    E_3 =  \begin{bmatrix} 1 & 0  & 0 & 0 \\ 0 & 0 & 0 & 1 \\ 0 & 0 & 1 & 0 \\ 0 & 1 & 0 & 0 \end{bmatrix}
  \]

  \vspace{0.5in}

\end{frame}
  
\begin{frame}{Elementary Row Operations via Elementary Matrices}
  \begin{theorem}
    If an elementary row operation is performed on an $m \times n$ matrix $A$, the resulting matrix can be written as $EA$, where the $m\times m$ matrix $E$ is an elementary matrix created by performing the same row operation on $I_m$.
  \end{theorem}
  
  \medskip
  
  \[
    E_2 =  \begin{bmatrix} 1 & 0  & 0 \\ 0 & 1 & -2 \\ 0 & 0 & 1 \end{bmatrix}_{3\times 3}
    \quad
    A = \begin{bmatrix} 1 & 2 & 3 & 4 \\ 5 & 6 & 7 & 8 \\ 9 & 10 & 11 & 12 \end{bmatrix}_{3 \times 4}
    \quad
    E_2 A =  \begin{bmatrix} 1 & 2 & 3 & 4 \\ \colorr{-13} & \colorr{-14} & \colorr{-15} & \colorr{-16} \\ 9 & 10 & 11 & 12 \end{bmatrix}_{3 \times 4}
  \]
\end{frame}


\begin{frame}{Elementary Matrices are Invertible}

  Every elementary matrix $E$ is invertible, and the inverse $E^{-1}$ is formed by applying the inverse elementary row operation to $I_m$.

  \[ 
    E_1 =  \begin{bmatrix} 1 & 0 \\ 0 & 5 \end{bmatrix}  \qquad
    E_2 =  \begin{bmatrix} 1 & 0  & 0 \\ 0 & 1 & -2 \\ 0 & 0 & 1 \end{bmatrix}  \qquad
    E_3 =  \begin{bmatrix} 1 & 0  & 0 & 0 \\ 0 & 0 & 0 & 1 \\ 0 & 0 & 1 & 0 \\ 0 & 1 & 0 & 0 \end{bmatrix}
  \]

  \bbox
    Write out matrices to represent  $E_1^{-1}$, $E^{-1}_2$, and $E^{-1}_3$.
  \ebox
\end{frame}


% \begin{frame}{The Product of Elementary Matrices}
%   % I think the point of this slide is confusing, and the slide doesn't fit into the flow of where we're going!!!
%   % Also, we never *use* the fact that elementary matrices are invertible.  They are a nice example that we can compute the inverse of more directly, but we don't need to do that for our general algorithm.
%   \smallskip
% 
%   Consider the matrix $A = \begin{bmatrix} 0 & 1 & -2 \\ 1 & 0 & 0 \\ 0 & 0 & 5 \end{bmatrix}$. Write $A$ as the product of three elementary matrices.
%   
%   \medskip
%   
% \begin{columns}[T]
% \column{0.4\tw}
% 
% %The matrix $A$ does three operations in the following order:
% \bb
% \ii \alert{Row 2 $=$ Row 2 $- 2 \cdot$ Row 3}
% \ii \colorb{Interchange Rows 1 and 2.}
% \ii \colorg{Scale Row 3 by $5$.}
% \ee
% 
% \column{0.6\tw}
% 
% \[ \alert{E_1 = \begin{bmatrix} 1 & 0 & 0 \\ 0 & 1 & -2 \\ 0 & 0 & 1 \end{bmatrix}} \ \ \ 
% \colorb{E_2 = \begin{bmatrix} 0 & 1 & 0 \\ 1 & 0 & 0 \\ 0 & 0 & 1 \end{bmatrix}} \ \ \ 
% \colorg{E_3 = \begin{bmatrix} 1 & 0 & 0 \\ 0 & 1 & 0 \\ 0 & 0 & 5 \end{bmatrix} }
%  \]
% 
% \end{columns}
% 
% 
% \[ A = \colorg{E_3}\colorb{E_2}\alert{E_1} = 
% \colorg{\begin{bmatrix} 1 & 0 & 0 \\ 0 & 1 & 0 \\ 0 & 0 & 5 \end{bmatrix}}  
% \colorb{\begin{bmatrix} 0 & 1 & 0 \\ 1 & 0 & 0 \\ 0 & 0 & 1 \end{bmatrix}}  
% \alert{\begin{bmatrix} 1 & 0 & 0 \\ 0 & 1 & -2 \\ 0 & 0 & 1 \end{bmatrix}} = 
% \begin{bmatrix} 0 & 1 & -2 \\ 1 & 0 & 0 \\ 0 & 0 & 5 \end{bmatrix}  \]
% 
% \[ A^{-1} = \alert{E_1^{-1}}\colorb{ E_2^{-1}} \colorg{E_3^{-1}} =  
% \alert{\begin{bmatrix} 1 & 0 & 0 \\ 0 & 1 & 2 \\ 0 & 0 & 1 \end{bmatrix} }
% \colorb{ \begin{bmatrix} 0 & 1 & 0 \\ 1 & 0 & 0 \\ 0 & 0 & 1 \end{bmatrix}}
% \colorg{  \begin{bmatrix} 1 & 0 & 0 \\ 0 & 1 & 0 \\ 0 & 0 & \frac{1}{5} \end{bmatrix}} = 
% \begin{bmatrix} 0 & 1 & 0 \\ 1 & 0 & \frac{2}{5} \\ 0 & 0 & \frac{1}{5} \end{bmatrix} \]
% 
% \end{frame}


\begin{frame}{RREF of an Invertible Matrix}
  \begin{theorem}
    An $n\times n$ matrix $A$ is invertible if and only if $A$ is row equivalent to $I_m$.
  \end{theorem}
  
  \begin{proof}
  \pause
  (\alert{$\Rightarrow$})
    Suppose that $A$ is invertible.
    Then $A\mathbf{x}=\mathbf{b}$ is solvable for every $\mathbf{b}$ in $\mathbb{R}^n$.
    Thus, there must be a pivot in every row of $A$.
    Since $A$ is square, the RREF of $A$ is $I_n$.
  \bigskip
    
  \pause
  (\alert{$\Leftarrow$})
    Suppose that the RREF of $A$ is $I_n$.
    Then there is a sequence $R_1,R_2,\dots,R_p$ of elementary row operations that transforms $A$ into $I_n$.
    Let $E_i$ be the elementary matrix corresponding to $R_i$.
    Then 
    \[
      \only<3>{E_p E_{p-1} \dots E_2 E_1 A = I_n}
      \only<4>{\underbrace{E_p E_{p-1} \dots E_2 E_1}_{\alert{A^{-1}}} A = I_n}
    \]
  \end{proof}
  
\end{frame}


\begin{frame}{An Algorithm for Finding $A^{-1}$}

Row reduce the augmented matrix $\begin{bmatrix} A \ \alert{|} \ I_n \end{bmatrix}$.
\bi
\ii  If RREF of $A$ is $I_n$, then RREF of $\begin{bmatrix} A \ \alert{|} \  I_n \end{bmatrix}$ is  $\begin{bmatrix} I_n \ \alert{|} \  A^{-1} \end{bmatrix}$.
\ii Otherwise, $A$ does not have an inverse.
\ei

\[ \begin{bmatrix} 
0 & 1 & -2 & \alert{|} & 1 & 0 & 0 \\ 
1 & 0 & 0 & \alert{|} & 0 & 1 & 0 \\ 
0 & 0 & 5 & \alert{|} & 0 & 0 & 1 \end{bmatrix} \rightarrow \begin{bmatrix} 
0 & 1 & -2 & \alert{|} & 1 & 0 & 0 \\ 
1 & 0 & 0 & \alert{|} & 0 & 1 & 0 \\ 
0 & 0 & 1 & \alert{|} & 0 & 0 & \frac{1}{5} \end{bmatrix} \rightarrow  \]

\[ \begin{bmatrix} 
0 & 1 & 0 & \alert{|} & 1 & 0 & \frac{2}{5} \\ 
1 & 0 & 0 & \alert{|} & 0 & 1 & 0 \\ 
0 & 0 & 1 & \alert{|} & 0 & 0 & \frac{1}{5} \end{bmatrix} \rightarrow  \begin{bmatrix} 
1 & 0 & 0 & \alert{|} & \colorb{0} & \colorb{1} & \colorb{0} \\ 
0 & 1 & 0 & \alert{|} & \colorb{1} & \colorb{0} & \colorb{\frac{2}{5}} \\ 
0 & 0 & 1 & \alert{|} & \colorb{0} & \colorb{0} & \colorb{\frac{1}{5}} \end{bmatrix} 
 \]

\end{frame}

\begin{frame}{Practice}

  If possible, find the inverse of the given matrix.

  \begin{tasks}(2)
    \task $\begin{bmatrix} 3 & 6 \\ 1 & 2 \end{bmatrix}$
    \task $\begin{bmatrix} -2 & 4 \\ -3 & 1 \end{bmatrix}$
  \end{tasks}

\end{frame}


\begin{frame}{Inverse of a $2\times 2$ Matrix}
  \smallskip
  
  Let $A=\begin{bmatrix} a & b \\ c & d\end{bmatrix}$.  How does $A$ \alert{fail} to be invertible?
  \pause If one row is a multiple of the other!
  
  \[
    \begin{bmatrix} a \\ b \end{bmatrix} = k \begin{bmatrix} c \\ d \end{bmatrix}
    \quad \colorr{\Leftrightarrow} \quad
    \begin{array}{rcl}
      a &=& kc \\
      b &=& kd
    \end{array}
    \quad \colorr{\Rightarrow} \quad
    \frac{a}{c} = k = \frac{b}{d}
    \quad \colorr{\Rightarrow} \quad
    ad=bc
    \quad \colorr{\Leftrightarrow} \quad
    ad-bc=0
  \]
  
  \pause
  (\emph{Need to consider some special cases:} $c$ or $d$ is $0$, and $(c,d)=0(a,b)$.)
  
  \pause
  \begin{theorem}
    $A$ is \alert{not invertible} if and only if $ad-bc=0$.
  \end{theorem}
  
  \vspace*{1em}
  
  \pause
  %\hspace*{8em}
  If $A$ is \alert{invertible}, then
  \smallskip
  
  $A^{-1}=\displaystyle\frac{1}{ad-bc} \begin{bmatrix} d & -b \\ -c & a \end{bmatrix}$.  %(Multiply it out!)
 
\end{frame}



\begin{frame}{Practice}

  If possible, find the inverse of the given matrix.

  \begin{tasks}(2)
    \task $\begin{bmatrix} 1 & 0 & 3 \\ 0 & -2 & -2\\ 1 & -3 & 1 \end{bmatrix}$
    \task $\begin{bmatrix} 2 & 0 & 1\\  0 & 1 & 3\\ -4 & 0 & -2 \end{bmatrix}$
  \end{tasks}

\end{frame}

% \begin{frame}[fragile]{Finding an Inverse Matrix}
% 
% \begin{columns}[T]
% 
% \column{0.5\tw}
% In Python, we can use \alert{Matrix.inv()} in the SymPy library.
% 
% \begin{lstlisting}
% from sympy import *
% A = Matrix([[1.0, 0.0, 3.0],
%             [0.0, -2.0, -2.0],
%             [1.0, -3.0, 1.0]])
% B = A.inv()
% print(B)
% \end{lstlisting}
% 
% \column{0.5\tw}
% 
% \bbox
% If $A = \begin{bmatrix} a & b \\ c & d \end{bmatrix}$, then
% \[ A^{-1} = \frac{1}{ad-bc} \begin{bmatrix} d & -b \\ -c & a \end{bmatrix} \]
% when $ad-bc \ne 0$.
% 
% If $ad -bc=0$, then $A$ is not invertible.
% \ebox
% 
% If $A = \begin{bmatrix} -2 & 4 \\ -3 & 1 \end{bmatrix}$, then we have
% 
% \[ \alert{A^{-1} = \frac{1}{10} \begin{bmatrix} 1 & -4 \\ 3 & -2 \end{bmatrix}} . \]
% 
% \end{columns}
% \end{frame}

\begin{frame}{Solving Systems of Linear Equations}

\begin{columns}[T]

\column{0.5\tw}

Solve the system of equations

\[ \begin{array}{rcl}
x_1+3x_3 &=&  2\\
-2x_2-2x_3 &=& 1\\
x_1-3x_2+x_3 &=& 0 \end{array} \]

\column{0.5\tw}

The equation above has matrix equation $A\mathbf{x} = \mathbf{b}$ given by

\[ \begin{bmatrix} 1 & 0 & 3 \\ 0 & -2 & -2\\ 1 & -3 & 1 \end{bmatrix} \mathbf{x} = \begin{bmatrix} 2 \\ 1 \\ 0 \end{bmatrix} . \]

\end{columns}
\bigskip

\pause
We compute the solution by $\mathbf{x} = A^{-1}\mathbf{b}$:

\[ \mathbf{x} = \begin{bmatrix}
4 & 4.5 & -3 \\
1 & 1 & -1 \\
-1 & -1.5 & 1 \end{bmatrix}  \begin{bmatrix} 2 \\ 1 \\ 0 \end{bmatrix}  = \begin{bmatrix}
12.5 \\ 3 \\ -3.5 \end{bmatrix} \]

\end{frame}


\begin{frame}{Solving Systems with Inverse Matrices}

\bb
\ii Write the system of linear equations as a matrix equation $A \mathbf{x} = \mathbf{b}$.
\ii Find $A^{-1}$ (if possible). 
\ii Multiply both sides of the equation in step (1) on the left by $A^{-1}$.
\ii The solution is $\mathbf{x} = A^{-1} \mathbf{b}$.
\ee

\bs

\bbox
\emph{Note.} If the number of equations and variables are not equal, then $A$ will not be a square matrix, and it will not be possible to find $A^{-1}$. Use row reduction on the augmented matrix instead.
\ebox
\bigskip

\pause
How does solving $A\mathbf{x}=\mathbf{b}$ with inverse matrices \alert{compare} to finding the RREF of the augmented matrix?

\end{frame}

\end{document}
