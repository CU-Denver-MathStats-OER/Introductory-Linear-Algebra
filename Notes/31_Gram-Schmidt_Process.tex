\documentclass[xcolor=dvipsnames,aspectratio=169,t]{beamer}
  % t means frames are vertically centered to the top
\usepackage{slides-header}
\title{The Gram--Schmidt Orthogonalization Process}

\begin{document}
\maketitle

\begin{frame}{Introduction}
  \smallskip

  \bi
  \ii We have seen that an orthogonal basis for a subspace $W$ of $\R^n$ is particularly nice.\smallskip
  \bi
  \ii For $\y$ in $\R^n$, we compute the \alert{orthogonal projection} of $\y$ onto $W$ by
  \[ \hat{\y} = \proj_W \y = c_1 \u_1 + c_2 \u_2 + \ldots + c_p \u_p \quad \mbox{with weights} \ c_i = \frac{\y \cdot \u_i}{\u_i \cdot \u_i} .\]
  \ei
  \ii When we choose an \alert{orthonormal basis}, the calculations are even simpler.
  \[ \hat{\y} = \proj_W \y = (UU^T) \y .\]
  \vspace*{-1em}
  \ii Recall $\hat{\y}$ is a useful vector to find as it is the \alert{best approximation} to $\y$ in the subspace $W$.
  \ei

  \pause
  \begin{example}
    Let $W  = \Span \left\{ \x_1 , \x_2 \right\}$ denote the subspace of $\R^3$ where $\x_1 = \begin{bmatrix} 3 \\ 6 \\ 0 \end{bmatrix}$ and $\x_2 = \begin{bmatrix} 1 \\ 2 \\ 2 \end{bmatrix}$. \\
    How can we find an \alert{orthogonal basis} for $W$?
  \end{example}
\end{frame}


\begin{frame}
  Let $W  = \Span \left\{ \x_1,\x_2 \right\}$ denote a subspace of $\R^3$ where
  $\x_1 = \begin{bmatrix} 3 \\ 6 \\ 0 \end{bmatrix}$ and
  $\x_2 = \begin{bmatrix} 1 \\ 2 \\ 2 \end{bmatrix}$.
  \vspace*{-.75em}
  
  How can we find an \alert{orthogonal basis} for $W$?
  \medskip

  \pause
  \begin{columns}[T]
  \column{0.19\tw}
  \vspace*{3em}
  
  \begin{tikzpicture}
    \coordinate (v1) at (1.75,0.5) {};
    \node[anchor=west] at (v1) {$\v_1=\x_1$};
    \draw[very thick,blue,-latex,inner sep=0pt] (0,0)--(v1);  % latex or stealth arrowhead
    \coordinate (x2) at (1,1) {};
    \node[anchor=south west] at (x2) {$\x_2$};
    \draw[very thick,black,-latex] (0,0)--(x2);
    \coordinate (p) at (1.18867924528302,0.339622641509434) {};
    \node[anchor=north west] at (p) {$\proj_{\v_1} \x_2$};
    \draw[thick,dotted] (1,1)--(p);
    \draw[very thick,red,-latex] (0,0)--(p);
    \coordinate (v2) at (-0.188679245283019,0.660377358490566) {};
    \node[anchor=south] at (v2) {$\v_2$};
    \draw[very thick,blue,-latex] (0,0)--(v2);
  \end{tikzpicture}

  \column{0.81\tw}
  We want orthogonal vectors $\v_1,\v_2$ such that
    $\Span \left\{ \x_1,\x_2 \right\} = \Span \left\{ \v_1,\v_2 \right\}$.
  \bi
  \ii Let $\v_1 = \x_1$.
  \pause
  \ii Let $\v_2$ be the orth proj of $\x_2$ onto the orth compl of $Y=\Span\{\v_1\}$.
  %Let $\v_2=\proj_{\Span\{\v_1\}^\perp} \x_2$.
  \smallskip
  
  By the \blue{Orth Decomp Thm}, $\x_2 = \proj_Y \x_2 + \z$, where $\z$ is in $Y^\perp$.
  
  \hspace*{7em} $\proj_Y \x_2 = \proj_{\v_1} \x_2 =  \left( \frac{ \x_2 \cdot \v_1}{\v_1 \cdot \v_1} \right) \v_1 = \frac{15}{45} {\small \begin{bmatrix} 3 \\ 6 \\ 0 \end{bmatrix} }= 
  {\small \begin{bmatrix} 1 \\ 2 \\ 0 \end{bmatrix}}$.
  \vspace*{-1em}
  
  \ii Hence $\v_2 = \z = \x_2 - \proj_{\v_1} \x_2 =
  {\small \begin{bmatrix} 1 \\ 2 \\ 2 \end{bmatrix}
        - \begin{bmatrix} 1 \\ 2 \\ 0 \end{bmatrix} =
           \begin{bmatrix} 0 \\ 0 \\ 2 \end{bmatrix}}$.
  \pause
  \ii Since $\v_2 = \x_2 - \proj_{\v_1} \x_2 = \x_2 + c \x_1$, $\v_2$ is in the subspace $W$.
  \pause
  \ii Thus $\left\{ \v_1 , \v_2 \right\} = \left\{ 
    {\small \begin{bmatrix} 3 \\ 6 \\ 0 \end{bmatrix} , 
            \begin{bmatrix} 0 \\ 0 \\ 2 \end{bmatrix} } \right\}$ 
    is an \alert{orthogonal basis} for $W$.
  \ei
  \end{columns}
\end{frame}


\begin{frame}{The Gram--Schmidt Orthogonalization Process}
  \smallskip
  
  Given a \alert{basis} $\left\{ \x_1, \x_2 , \ldots , \x_p \right\}$ for a nonzero subspace $W$ of $\R^n$, define
  \begin{align*}
  \v_1 &= \x_1  & Y_1 &= \Span\{\v_1\} \\
  \onslide<2->{
  \v_2 &= \proj_{Y_1^\perp} \x_2 = \x_2 - \proj_{Y_1} \x_2 = \x_2 - \left( \textstyle \frac{ \x_2 \cdot \v_1}{\v_1 \cdot \v_1} \right) \v_1  & Y_2 &= \Span\{\v_1,\v_2\}\\
  }
  \onslide<3->{
  \v_3 & = \proj_{Y_2^\perp} \x_3 = \x_3 - \proj_{Y_2} \x_3
  = \x_3 - \left( \textstyle \frac{ \x_3 \cdot \v_1}{\v_1 \cdot \v_1} \right) \v_1 - \left( \textstyle \frac{ \x_3 \cdot \v_2}{\v_2 \cdot \v_2} \right) \v_2 
  & Y_3 &= \Span\{\v_1,\v_2,\v_3\} \\
  }
  \onslide<4->{
  & \vdots \\
  \v_p & = \proj_{Y_{p-1}^\perp} \x_p = \x_p - \proj_{Y_{p-1}} \x_p \\
  &= \x_p - \left( \textstyle \frac{ \x_p \cdot \v_1}{\v_1 \cdot \v_1} \right) \v_1 - \left( \textstyle \frac{ \x_p \cdot \v_2}{\v_2 \cdot \v_2} \right) \v_2 - \ldots - \left( \textstyle \frac{ \x_p \cdot \v_{p-1}}{\v_{p-1} \cdot \v_{p-1}} \right) \v_{p-1}.
  & Y_p &= \Span\{\v_1,\v_2,\v_3,\ldots,\v_p\}
  }
  \end{align*}
  \vspace*{-1em}

  \onslide<5->{
  \begin{theorem}
    Then $\left\{ \v_1 , \v_2 , \ldots , \v_p \right\}$ is an \alert{orthogonal basis} for $W$.
    \onslide<6->{
    In addition,
    \vspace*{-.75em}
    
    \[ \Span \left\{ \x_1 , \x_2 , \ldots , \x_k \right\} = Y_k = \Span \left\{ \v_1 , \v_2 , \ldots , \v_k \right\}  \quad \mbox{for } 1 \leq k \leq p. \]%
    \vspace*{-1.3em}% I don't know why this is needed.  Something about the overlay messing up the spacing.
    }%
  \end{theorem}%
  }%
\end{frame}


\begin{frame}{Example}
  Find an \alert{orthogonal basis} for the subspace $W = \Span \left\{ 
  {\small \begin{bmatrix} 1 \\ -4 \\ 0 \\ 1 \end{bmatrix},
          \begin{bmatrix} 7 \\ -7 \\ -4 \\ 1 \end{bmatrix} } \right\}$ in $\R^4$.

  \vspace{-0.2in}

  \pause
  We have  $\v_1 = \x_1 =  \begin{bmatrix} 1 \\ -4 \\ 0 \\ 1 \end{bmatrix}$. 
  \vspace{-0.2em}

  \pause
  Then we find $ \v_2 = \x_2 - \left( \frac{ \x_2 \cdot \v_1}{\v_1 \cdot \v_1} \right) \v_1 = \begin{bmatrix} 7 \\ -7 \\ -4 \\ 1 \end{bmatrix} - \left( \displaystyle \frac{36}{18} \right)  \begin{bmatrix} 1 \\ -4 \\ 0 \\ 1 \end{bmatrix} = \begin{bmatrix} 5 \\ 1 \\ -4 \\ -1 \end{bmatrix}.$
  \bigskip

  Thus an \alert{orthogonal basis} for $W$ is  $\left\{  \begin{bmatrix} 1 \\ -4 \\ 0 \\ 1 \end{bmatrix} ,  \begin{bmatrix} 5 \\ 1 \\ -4 \\ -1 \end{bmatrix} \right\}$.
\end{frame}


\begin{frame}{Example}
  \smallskip
  
  Consider the matrix $A = \small \begin{bmatrix} 3 & -5 & 1 \\ 1 & 1 & 1 \\ -1 & 5 & -2 \\ 3 & -7 & 8 \end{bmatrix}$.
  Find an \alert{orthogonal basis} for the column space of $A$.
  \bigskip

  \pause
  First we find a basis for the column space by checking the reduced row echelon form of $A$:

  \[ \small \begin{bmatrix} 3 & -5 & 1 \\ 1 & 1 & 1 \\ -1 & 5 & -2 \\ 3 & -7 & 8 \end{bmatrix} \xrightarrow{\text{RREF}} \begin{bmatrix} 1 & 0 & 0 \\ 0 & 1 & 0\\ 0 & 0 &  1 \\ 0 & 0 & 0 \end{bmatrix}, \mbox{ so a basis for $\Col A$ is}
  \left\{ \begin{bmatrix} 3\\ 1 \\ -1 \\ 3 \end{bmatrix} , \begin{bmatrix} -5 \\ 1 \\ 5 \\ -7 \end{bmatrix}, \begin{bmatrix} 1 \\ 1 \\ -2 \\ 8 \end{bmatrix} \right\}. \]
  \vspace*{2em}

  Next we use the Gram--Schmidt Process to convert the basis into an \alert{orthogonal} basis.
\end{frame}


\begin{frame}{Example, continued}

  \blue{ $\Col A = \Span \left\{ \small \begin{bmatrix} 3\\ 1 \\ -1 \\ 3 \end{bmatrix} , \begin{bmatrix} -5 \\ 1 \\ 5 \\ -7 \end{bmatrix}, \begin{bmatrix} 1 \\ 1 \\ -2 \\ 8 \end{bmatrix} \right\}$}.
  We first define \alert{$\v_1 =  \begin{bmatrix} 3\\ 1 \\ -1 \\ 3 \end{bmatrix}$}.
  \medskip

  \pause
  Then we continue with Gram--Schmidt process:
  \begin{align*}
  \colorg{\v_2} & = \colorb{\x_2} - \left( \frac{ \colorb{\x_2} \cdot \alert{\v_1}}{\alert{\v_1} \cdot \alert{\v_1}} \right) \alert{\v_1} = \colorb{\begin{bmatrix} -5 \\ 1 \\ 5 \\ -7 \end{bmatrix}} -  \left( \frac{-40}{20} \right) \alert{\begin{bmatrix} 3\\ 1 \\ -1 \\ 3 \end{bmatrix}} = \colorg{\begin{bmatrix}  1 \\ 3 \\ 3 \\ -1 \end{bmatrix}} \\[1em]
  \onslide<3->{
  \v_3 &= \colorb{\x_3} - \left( \frac{ \colorb{\x_3} \cdot \alert{\v_1}}{\alert{\v_1} \cdot \alert{\v_1}} \right) \alert{\v_1} - \left( \frac{ \colorb{\x_3} \cdot \colorg{\v_2}}{\colorg{\v_2} \cdot \colorg{\v_2}} \right) \colorg{\v_2} = \colorb{\begin{bmatrix} 1 \\ 1 \\ -2 \\ 8 \end{bmatrix}} - \left( \frac{30}{20} \right) \alert{\begin{bmatrix} 3\\ 1 \\ -1 \\ 3 \end{bmatrix}} - \frac{-10}{20} \colorg{\begin{bmatrix}  1 \\ 3 \\ 3 \\ -1 \end{bmatrix}} = \begin{bmatrix} -3 \\ 1 \\ 1 \\ 3 \end{bmatrix}
  }
  \end{align*}
\end{frame}


\begin{frame}{Normalizing the Basis}
  \smallskip

  For the $4 \times 3$ matrix $A = \begin{bmatrix} 3 & -5 & 1 \\ 1 & 1 & 1 \\ -1 & 5 & -2 \\ 3 & -7 & 8 \end{bmatrix}$, we therefore have
  \medskip

  \bi
  %\ii An initial basis for $\Col A$ is  $\left\{ \begin{bmatrix} 3\\ 1 \\ -1 \\ 3 \end{bmatrix} , \begin{bmatrix} -5 \\ 1 \\ 5 \\ -7 \end{bmatrix}, \begin{bmatrix} 1 \\ 1 \\ -2 \\ 8 \end{bmatrix} \right\}$

  \ii An \colorb{orthogonal basis} for $\Col A$ is  \colorb{$\left\{ \begin{bmatrix} 3\\ 1 \\ -1 \\ 3 \end{bmatrix} , \begin{bmatrix} 1 \\ 3 \\ 3 \\ -1 \end{bmatrix}, \begin{bmatrix} -3 \\ 1 \\ 1 \\ 3 \end{bmatrix} \right\}$}.

  \bs

  \pause
  \ii An \alert{orthonormal basis} for $\Col A$ is  \alert{$\left\{ \begin{bmatrix} 3/\sqrt{20}\\ 1/\sqrt{20} \\ -1/\sqrt{20} \\ 3/\sqrt{20} \end{bmatrix} , \begin{bmatrix} 1/\sqrt{20} \\ 3/\sqrt{20} \\ 3/\sqrt{20} \\ -1/\sqrt{20} \end{bmatrix}, \begin{bmatrix} -3/\sqrt{20} \\ 1/\sqrt{20} \\ 1/\sqrt{20} \\ 3/\sqrt{20} \end{bmatrix} \right\}$}.
  \ei
\end{frame}

\end{document}  % We are skipping QR factorization.

\begin{frame}{The $QR$ Factorization}
  Let 
  $A = \small \begin{bmatrix} 3 & -5 & 1 \\ 1 & 1 & 1 \\ -1 & 5 & -2 \\ 3 & -7 & 8 \end{bmatrix}$.
  We can construct a matrix $Q$ whose columns form an orthonormal basis for $\Col A$.

  \vspace{-0.1in}

  \alert{\[ Q =  \begin{bmatrix} 3/\sqrt{20} & 1/\sqrt{20} & -3/\sqrt{20} \\
  1/\sqrt{20} & 2/\sqrt{20} & 1/\sqrt{20} \\
  -1/\sqrt{20} & 3/\sqrt{20} & 1/\sqrt{20} \\
  3/\sqrt{20} & -1/\sqrt{20} & 3/\sqrt{20} \end{bmatrix} \]}

  \vspace{-0.1in}

  Next we define

  \vspace{-0.1in}

  {\small \[ \colorb{R} = Q^T A = \begin{bmatrix} 3/\sqrt{20} & 1/\sqrt{20} & -1/\sqrt{20} & 3/\sqrt{20} \\
  1/\sqrt{20} & 2/\sqrt{20} & 3/\sqrt{20} & -1/\sqrt{20} \\
  -3/\sqrt{20} & 1/\sqrt{20} & 1/\sqrt{20} & 3/\sqrt{20} \end{bmatrix}  \begin{bmatrix} 3 & -5 & 1 \\ 1 & 1 & 1 \\ -1 & 5 & -2 \\ 3 & -7 & 8 \end{bmatrix} = \colorb{\begin{bmatrix} \sqrt{20} & -2\sqrt{20} & 3\sqrt{20}/2 \\
  0 & \sqrt{20} & -\sqrt{20}/2 \\ 0 & 0 & \sqrt{20} \end{bmatrix}} \] }
\end{frame}

\begin{frame}

In our previous example, $A = \begin{bmatrix} 3 & -5 & 1 \\ 1 & 1 & 1 \\ -1 & 5 & -2 \\ 3 & -7 & 8 \end{bmatrix}$.  Using Gram--Schmidt, we construct a matrix \alert{$Q$} whose columns form an orthonormal basis for $\Col A$ and \colorb{$R = Q^TA$}.

\[ \alert{Q =  \begin{bmatrix} 3/\sqrt{20} & 1/\sqrt{20} & -3/\sqrt{20} \\
1/\sqrt{20} & 2/\sqrt{20} & 1/\sqrt{20} \\
-1/\sqrt{20} & 3/\sqrt{20} & 1/\sqrt{20} \\
3/\sqrt{20} & -1/\sqrt{20} & 3/\sqrt{20} \end{bmatrix}} \quad \mbox{and} \quad
 \colorb{R = \begin{bmatrix} \sqrt{20} & -2\sqrt{20} & 3\sqrt{20}/2 \\
0 & \sqrt{20} & -\sqrt{20}/2 \\ 0 & 0 & \sqrt{20} \end{bmatrix}} \] 

Taking the product of matrices $Q$ and $R$ gives
$\alert{Q}\colorb{R} = \alert{Q} \colorb{Q^TA} = (\colorr{Q}\colorb{Q^T}) A$.
The matrix $Q Q^T$ orthogonally projects vectors onto $\Col A$.
Since each column of $A$ is in $\Col A$, $Q Q^T A = A$.
\bigskip

\pause
In other words, given a matrix $A$ with linearly independent columns, we can express $A=\alert{Q}\colorb{R}$ as a product of two matrices, \alert{$Q$} (whose columns form an \alert{orthonormal basis for $\Col A$}) and an \colorb{upper-triangular} matrix \colorb{$R$} with positive entries along the main diagonal.
\end{frame}


\begin{frame}{The $QR$ Factorization}

\begin{theorem}[$QR$ Factorization]
If $A$ is an $m \times n$ matrix \colorg{with linearly independent columns}, then $A$ can be factored as $A = QR$ where:
\bi
\ii \alert{$Q$} is an $m \times n$ matrix whose \alert{columns form an orthonormal basis for $\Col A$}, and 
\ii \colorb{$R$} is an \colorb{$n \times n$ upper-triangular matrix} with positive entries along the main diagonal \\(so \colorb{$R$ is invertible}).
\ei
\end{theorem}

\pause
We can compute $\colorb{R}=\colorr{Q^T} A$.  Why is $\colorb{R}$ upper-triangular?
\bigskip

Since $A=\colorr{Q}\colorb{R}$, we can write $\mathbf{a}_j=\colorr{Q} \colorb{\mathbf{r}_j}$ for $1\le j \le n$.

Since
$\Span \left\{ \a_1 , \ldots , \a_k \right\} =  \Span \left\{ \mathbf{q}_1 , \ldots , \mathbf{q}_k \right\}$ for $1 \leq k \leq n$,
$\mathbf{a}_j$ is in $\Span\left\{ \mathbf{q}_1 , \ldots , \mathbf{q}_j \right\}$.

Thus $\colorb{\mathbf{r}_j}$ can be chosen such that $\colorb{r_{ij}}=0$ for $i>j$.
\bigskip

The fact that $\colorb{R}$ has nonzero entries along the main diagonal comes from the fact that $\colorb{R}$ is invertible (which follows from the fact that $A$ has linearly independent columns).

%This factorization is widely used in computer algorithms for various computations, such as linear regression, where we want to find the best fitting model corresponding to a dataset.
\end{frame}

\begin{frame}{Steps for Finding a $QR$ Factorization}

{\small Find the $QR$ factorization for $\colorg{A = \begin{bmatrix} 1 & 0 & 0 \\ 1 & 1 & 0 \\ 1 & 1 & 1 \\ 1 & 1 & 1 \end{bmatrix}}$.
\pause
$\Col A = \Span \left\{ \begin{bmatrix} 1 \\ 1 \\ 1 \\ 1 \end{bmatrix}, \begin{bmatrix} 0 \\ 1 \\ 1 \\ 1 \end{bmatrix}, \begin{bmatrix} 0 \\ 0 \\ 1 \\ 1 \end{bmatrix} \right\}$.}
\medskip

\bb
\ii Find an orthogonal basis using the Gram--Schmidt Process.

{\small \[ \Col A = \Span \left\{ 
\begin{bmatrix} 1 \\ 1 \\ 1\\ 1 \end{bmatrix},
\begin{bmatrix} -3/4 \\ 1/4 \\ 1/4 \\ 1/4 \end{bmatrix},
\begin{bmatrix} 0 \\ -2/3 \\ 1/3 \\ 1/3 \end{bmatrix} \right\}. 
\quad
\onslide<3->{
\alert{Q = \begin{bmatrix} 1/2 & -3/\sqrt{12} & 0 \\
1/2 & 1/\sqrt{12} & -2/\sqrt{6} \\
1/2 & 1/\sqrt{12} & 1/\sqrt{6} \\
1/2 & 1/\sqrt{12} & 1/\sqrt{6} \end{bmatrix}}.} \]}

\pause
\ii Normalize each vector to find \alert{$Q$}.

\pause
\ii Find {\small $\colorb{R} = \alert{Q}^T\colorg{A} = \alert{\begin{bmatrix} 1/2 & 1/2 & 1/2 & 1/2 \\
-3/\sqrt{12} & 1/\sqrt{12} & 1/\sqrt{12} & 1/\sqrt{12} \\
 0 & -2/\sqrt{6} &  1/\sqrt{6}  & 1/\sqrt{6}    \end{bmatrix}}\colorg{\begin{bmatrix} 1 & 0 & 0 \\ 1 & 1 & 0 \\ 1 & 1 & 1 \\ 1 & 1 & 1 \end{bmatrix}} = \colorb{\begin{bmatrix} 2 & 3/2 & 1 \\ 0 & 3/\sqrt{12} & 2/\sqrt{12} \\ 0 & 0 & 2/\sqrt{6} \end{bmatrix}}.$}
\ee

\end{frame}

\end{document}

