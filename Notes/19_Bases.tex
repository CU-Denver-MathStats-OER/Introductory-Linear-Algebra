\documentclass[xcolor=dvipsnames,aspectratio=169,t]{beamer}
  % t means frames are vertically centered to the top
\usepackage{slides-header}
\title{Linear Independence and Bases}

\begin{document}
\maketitle

\begin{frame}{Finding an Efficient Set of Vectors that Span $V$}

Consider the vector space $V = \mathbb{R}^3$ with the usual operations, and consider the following set of vectors:

\[ \left\{ \begin{bmatrix} 1 \\ 0 \\ 0 \end{bmatrix} ,  \begin{bmatrix} 0 \\ 1 \\ 0 \end{bmatrix} , \begin{bmatrix} 0 \\ 0 \\ 1 \end{bmatrix} , \begin{bmatrix} 1 \\ 0 \\ -1 \end{bmatrix}, \begin{bmatrix} 0 \\ -2 \\ 5 \end{bmatrix}  \right\} . \]


\bb
\ii Do the set of vectors above span all of $\mathbb{R}^3$? \ms

\pause
\colorb{YES. Given any vector $\mathbf{b} = \begin{bmatrix} b_1 \\ b_2 \\ b_3 \end{bmatrix}$ in $\mathbb{R}^3$, we have
$\dsty \mathbf{b} = b_1 \begin{bmatrix} 1 \\ 0 \\ 0 \end{bmatrix}  + b_2   \begin{bmatrix} 0 \\ 1 \\ 0 \end{bmatrix}  + b_3  \begin{bmatrix} 0 \\ 0 \\ 1 \end{bmatrix}$, 
so $\mathbf{b}$ is in the span of the set of vectors above.}

\pause
\ii Do we need all five vectors in the set to span all of $\mathbb{R}^3$?
\ee

%\alert{NO. We only need the three standard column vectors to span $\mathbb{R}^3$. Notice  
%\[ \begin{bmatrix} 0 \\ -2 \\ 5 \end{bmatrix}  =  -2   \begin{bmatrix} 0 \\ 1 \\ 0 \end{bmatrix}  + 5  \begin{bmatrix} 0 \\ 0 \\ 1 \end{bmatrix} \quad \mbox{and} \quad
%\begin{bmatrix} 1 \\ 0 \\ -1 \end{bmatrix}  =    \begin{bmatrix} 1 \\ 0 \\ 0 \end{bmatrix}  -  \begin{bmatrix} 0 \\ 0 \\ 1 \end{bmatrix}.\]

%\bi
%\ii Removing these two vectors from the set above does not change the span.
%\ii Removing one of the standard column vectors would affect the span of the set.
%\ii We need a minimum of three vectors to span all of $\mathbb{R}^3$.
%\ei}

\end{frame} 

\begin{frame}{Finding an Efficient Set of Vectors that Span $V$}


\begin{columns}[T]

\column{0.425\tw}

Consider the vector space $V = \mathbb{R}^3$ with the usual operations, and consider the following set of vectors:

\[ \left\{ \begin{bmatrix} 1 \\ 0 \\ 0 \end{bmatrix} ,  \begin{bmatrix} 0 \\ 1 \\ 0 \end{bmatrix} , \begin{bmatrix} 0 \\ 0 \\ 1 \end{bmatrix} , \begin{bmatrix} 1 \\ 0 \\ -1 \end{bmatrix}, \begin{bmatrix} 0 \\ -2 \\ 5 \end{bmatrix}  \right\} . \]


\bb
\ii Do the set of vectors above span all of $\mathbb{R}^3$?  \colorb{YES} \ms 

\ii Do we need all five vectors in the set to span all of $\mathbb{R}^3$? \ms
\ee

\column{0.575\tw}

\colorb{NO. We only need the three standard column vectors to span $\mathbb{R}^3$. Notice  
\[ \begin{bmatrix} 0 \\ -2 \\ 5 \end{bmatrix}  =  -2   \begin{bmatrix} 0 \\ 1 \\ 0 \end{bmatrix}  + 5  \begin{bmatrix} 0 \\ 0 \\ 1 \end{bmatrix} \quad \mbox{and} \quad
\begin{bmatrix} 1 \\ 0 \\ -1 \end{bmatrix}  =    \begin{bmatrix} 1 \\ 0 \\ 0 \end{bmatrix}  -  \begin{bmatrix} 0 \\ 0 \\ 1 \end{bmatrix}.\]}

\pause
\bi
\ii Removing these two vectors from the set above does not change the span.
\ii Then also removing one of the standard column vectors would affect the span of the set.
\ii We need a minimum of three vectors to span all of $\mathbb{R}^3$.
\ei

\end{columns}

\end{frame} 

\begin{frame}{Revisiting Linear Independence}
  \bbox
  Let $V$ denote a vector space, and consider the set of vectors $S = \left\{ \mathbf{v}_1 ,  \mathbf{v}_2 , \ldots ,  \mathbf{v}_p \right\}$.%
  \bi
  \ii The set $S$ of vectors is \alert{linearly independent} if the vector equation $c_1  \mathbf{v}_1 + c_2  \mathbf{v}_2 + \ldots + c_p  \mathbf{v}_p = \mathbf{0}$ has \alert{only the trivial solution}. \ms
  \ii The set $S$ of vectors is \alert{linearly dependent} if the vector equation above has a \alert{non-trivial solution}.
  \ei
  \ebox

  \pause
  \begin{theorem}
  A set $\left\{ \mathbf{v}_1 ,  \mathbf{v}_2 , \ldots ,  \mathbf{v}_p \right\}$ of vectors is \alert{linearly dependent} if and only if
  \smallskip

  \qquad some $\mathbf{v}_j$ is a linear combination of the other vectors.
  \end{theorem}
\end{frame}

\begin{frame}{Basis of a Vector Space}

\bbox
Let $H$ denote a subspace of a vector space $V$. A set of vectors $\mathcal{B}$ in $V$ is \alert{basis for $H$} if:
\bb[(i)]
\ii $\mathcal{B}$ is a \alert{linearly independent set}, and
\ii the subspace spanned by $\mathcal{B}$ equals $H$. Namely \alert{$\mbox{Span } \mathcal{B} = H$}.
\ee
\ebox

\pause
\begin{columns}[T]

\column{0.42\tw}

\[ \left\{ \begin{bmatrix} 1 \\ 0 \\ 0 \end{bmatrix} ,  \begin{bmatrix} 0 \\ 1 \\ 0 \end{bmatrix} , \begin{bmatrix} 0 \\ 0 \\ 1 \end{bmatrix} , \begin{bmatrix} 1 \\ 0 \\ -1 \end{bmatrix}, \begin{bmatrix} 0 \\ -2 \\ 5 \end{bmatrix}  \right\}  \] \ss

\begin{center}\alert{NOT A BASIS} \end{center}

\column{0.29\tw}

\onslide*<2>{

\vspace{1em}

\[ \left\{ \begin{bmatrix} 1 \\ 0 \\ 0 \end{bmatrix} ,  \begin{bmatrix} 0 \\ 1 \\ 0 \end{bmatrix}  \right\}  \] \ss

\begin{center}\alert{NOT A BASIS} \end{center}
}
\onslide*<3->{
Note that there is \alert{not} a unique basis.
\[ \left\{ \begin{bmatrix} 1 \\ 0 \\ 0 \end{bmatrix} ,  \begin{bmatrix} 1 \\ 0 \\ -1 \end{bmatrix},
\begin{bmatrix} 0 \\ -2 \\ 5 \end{bmatrix} 
\right\} \]
\begin{center} \colorb{ALSO A BASIS} \end{center}
}

\column{0.29\tw}

\[ \left\{ \begin{bmatrix} 1 \\ 0 \\ 0 \end{bmatrix} ,  \begin{bmatrix} 0 \\ 1 \\ 0 \end{bmatrix} , \begin{bmatrix} 0 \\ 0 \\ 1 \end{bmatrix} \right\} \] \ss

\begin{center} \colorb{WE HAVE A WINNER!} \end{center}

\end{columns}


\end{frame}


\begin{frame}{Spanning Set Theorem}

\begin{theorem}
Let $S = \left\{ \mathbf{v}_1 ,  \mathbf{v}_2 , \ldots ,  \mathbf{v}_p \right\}$ be a set in a vector space $V$, and let $H= \mbox{Span}\left\{ \mathbf{v}_1 ,  \mathbf{v}_2 , \ldots ,  \mathbf{v}_p \right\}$ (which is a subspace of $V$).
\bi
\ii If one of the vectors in $S$ is a linear combination of the remaining vectors in $S$, then after removing that vector from the set, the remaining vectors will still span $H$.
\ii If $H \ne \left\{ \mathbf{0} \right\}$, some nonempty subset of $S$ is a basis for $H$.
\ei
\end{theorem}

  \begin{example}
    \[ H = \mbox{Span} \left\{ \begin{bmatrix} 1 \\ 0 \\ 0 \end{bmatrix} ,  \begin{bmatrix} 0 \\ 1 \\ 0 \end{bmatrix} , \begin{bmatrix} 0 \\ 0 \\ 1 \end{bmatrix} , \begin{bmatrix} 1 \\ 0 \\ -1 \end{bmatrix}, \begin{bmatrix} 0 \\ -2 \\ 5 \end{bmatrix}  \right\}  \] 
  \end{example}
\end{frame}


\begin{frame}{Standard Basis for $\mathbb{R}^n$}
  \bigskip

  We saw that  $\left\{ \begin{bmatrix} 1 \\ 0 \\ 0 \end{bmatrix} ,  \begin{bmatrix} 0 \\ 1 \\ 0 \end{bmatrix} , \begin{bmatrix} 0 \\ 0 \\ 1 \end{bmatrix} \right\} = \left\{ \mathbf{e}_1 ,  \mathbf{e}_2 ,  \mathbf{e}_3 \right\}$ is a basis for $\mathbb{R}^3$, and probably this is the most natural basis.  We extend this basis to higher dimensions.
  \bigskip

  \bbox
  The set of standard column vectors $\left\{  \mathbf{e}_1 ,  \mathbf{e}_2 , \ldots ,  \mathbf{e}_n \right\}$ that forms a basis for $\mathbb{R}^n$ is called the \alert{standard basis for $\mathbb{R}^n$.}
  \ebox
\end{frame}


\begin{frame}{Examples in $\mathbb{R}^3$}

Determine whether the given set of vectors is a basis for $\mathbb{R}^3$. %Using the Inverse Matrix Theorem, we have many ways to investigate. The set will be a basis if: 

%\bi
%\ii The matrix of column vectors $\begin{bmatrix} \mathbf{v}_1 & \mathbf{v}_2 & \mathbf{v}_3 \end{bmatrix}$ is invertible.
%\ii The matrix of column vectors $\begin{bmatrix} \mathbf{v}_1 & \mathbf{v}_2 & \mathbf{v}_3 \end{bmatrix}$ has a pivot in every column.
%\ii The determinant of the matrix of column vectors $\begin{bmatrix} \mathbf{v}_1 & \mathbf{v}_2 & \mathbf{v}_3 \end{bmatrix}$ is nonzero.
%\ei}

\begin{columns}[T]

\column{0.5\tw}

$\left\{ \begin{bmatrix} 3 \\ 0 \\ -6 \end{bmatrix} , \begin{bmatrix} -4 \\ 1 \\ 7 \end{bmatrix} , \begin{bmatrix} -2 \\ 1 \\ 5 \end{bmatrix} \right\}$   \bs

\pause
$\dsty \begin{bmatrix} 3 & -4 & -2 \\ 0 & 1 & 1 \\ -6 & 7 & 7 \end{bmatrix} \sim \begin{bmatrix} 1 & 0 & 0 \\ 0 & 1 & 0 \\ 0 & 0 & 1 \end{bmatrix}$ \ms

$\dsty \det \begin{bmatrix} 3 & -4 & -2 \\ 0 & 1 & 1 \\ -6 & 7 & 7 \end{bmatrix} = 6$


\column{0.5\tw}
\pause
$\left\{ \begin{bmatrix} 0 \\ 2 \\ -1 \end{bmatrix} , \begin{bmatrix} 2 \\ 2 \\ 0 \end{bmatrix} , \begin{bmatrix} 6 \\ 16 \\ -5 \end{bmatrix} \right\}$ \bs

\pause
{\small
$\begin{bmatrix} 0 & 2 & 6 \\ 2 & 2 & 16 \\ -1 & 0 & -5 \end{bmatrix} \sim \begin{bmatrix} 1 & 0 & \colorb{5} \\ 0 & 1 & \alert{3} \\ 0 & 0 & 0 \end{bmatrix}$ \ms

$\begin{bmatrix}x_1 \\ x_2 \\ x_3 \end{bmatrix} = 
\begin{bmatrix} -5 x_3 \\ -3 x_3 \\ x_3 \end{bmatrix} = 
x_3 \begin{bmatrix} -5 \\ -3 \\ 1 \end{bmatrix}$.
}
\medskip

Setting $x_3=-1$ and rearranging, we see that \ms

$\dsty \begin{bmatrix} 6 \\ 16 \\ -5 \end{bmatrix} = \colorb{5}  \begin{bmatrix} 0 \\ 2 \\ -1 \end{bmatrix} + \alert{3} \begin{bmatrix} 2 \\ 2 \\ 0 \end{bmatrix}$

\end{columns}

\end{frame}

\begin{frame}{Bases for More Abstract Vector Spaces}

\bb
\ii Find a basis for $\mbox{Mat}_{2 \times 2}$ the vector space of all $2 \times 2$ matrices with the usual operations. \ms

\colorb{\small
\[ \begin{bmatrix} a & b \\ c & d \end{bmatrix} = a \begin{bmatrix} 1 & 0 \\ 0 & 0 \end{bmatrix} + b \begin{bmatrix} 0 & 1 \\ 0 & 0 \end{bmatrix} + c \begin{bmatrix} 0 & 0 \\ 1 & 0 \end{bmatrix} + d \begin{bmatrix} 0 & 0 \\ 0 & 1 \end{bmatrix} \]}


\colorb{Thus we have $\dsty \mathcal{B} = \left\{ \begin{bmatrix} 1 & 0 \\ 0 & 0 \end{bmatrix} , \begin{bmatrix} 0 & 1 \\ 0 & 0 \end{bmatrix} ,  \begin{bmatrix} 0 & 0 \\ 1 & 0 \end{bmatrix} ,  \begin{bmatrix} 0 & 0 \\ 0 & 1 \end{bmatrix} \right\}$.}

\vspace*{2em}

\pause
\ii Find a basis for $\mathbb{P}_4$, the vector space of all polynomials of degree at most 4 with the usual operations. \ms

\colorb{An arbitrary vector in $\mathbb{P}_4$ is given by $p(t) = a_4t^4 + a_3t^3 + a_2t^2+a_1t + a_0$. Thus we have $\mathcal{B} = \left\{ 1, t, t^2, t^3, t^4 \right\}$.}

\ee
\end{frame}


\begin{frame}{Polynomial Vector Space Example 1}
  \medskip

Is \alert{$\left\{ 2t^2+1, t^2-1, t+2 \right\}$} a basis for \alert{$\mathbb{P}_2$}? ($\mathbb{P}_2=$polynomials of degree at most $2$) \vspace{0.25in}

\pause
We solve $c_1 ( 2t^2+1) + c_2 (t^2-1) + c_3(t+2)= 0$. This gives

\[ (2c_1 + c_2)t^2 + c_3t + (c_1-c_2) = 0 \]

From the linear and constant terms respectively, we see that $c_3=0$ and $c_1 = c_2$. Then looking at the coefficient in front of the quadratic term, we have $2c_1 + c_2 = 2c_1 + c_1 = 3c_1 =0$. Thus the only solution is the trivial solution, $c_1=c_2=c_3=0$. The set is linearly independent. \ms

Note the standard basis for $\mathbb{P}_2$ is $\left\{ t^2, t, 1 \right\}$, which contains three vectors. We have a linearly independent set of the same number of vectors, thus this set is a basis for $\mathbb{P}_2$.


\end{frame}

\begin{frame}{Polynomial Vector Space Example 2}
\medskip

Is \alert{$\left\{ 2t^2+t, t+1, 2t^2-1 \right\}$} a basis for \alert{$\mathbb{P}_2$}?  \vspace{0.5in}

\pause
Notice that we have $2t^2-1 = (2t^2+t) - (t+1)$, or $\mathbf{v}_3 = \mathbf{v}_1 - \mathbf{v}_2$. Since the set is linearly dependent, it cannot be a basis for $\mathbb{P}_2$.

\end{frame}

\begin{frame}{Matrix Vector Space Example}
  \bigskip

  Is \alert{$\left\{ \begin{bmatrix} 1 & 0 \\ 0 & 0 \end{bmatrix} , \begin{bmatrix} 0 & 1 \\ -1 & 0 \end{bmatrix} ,  \begin{bmatrix} 0 & 0 \\ 0 & 1 \end{bmatrix} ,  \begin{bmatrix} 2 & -4 \\ 4 & 0 \end{bmatrix} \right\}$} a basis for $\mbox{Mat}_{2 \times 2}$?
  \vspace{0.3in}

  \pause
  Notice that we have 

  \[  \begin{bmatrix} 2 & -4 \\ 4 & 0 \end{bmatrix} = 2  \begin{bmatrix} 1 & 0 \\ 0 & 0 \end{bmatrix} - 4 \begin{bmatrix} 0 & 1 \\ -1 & 0 \end{bmatrix} . \]
  \medskip

  Since one matrix can be written as a linear combination of other matrices in the set, this is a \alert{linearly dependent set}, and it cannot be a basis for $\mbox{Mat}_{2 \times 2}$.
\end{frame}

%%%%%%%%%%%%%%%%%%%%%%%%%%%%%%%%%%%%%%%%%%%%%%%%%%%%%%%%%%%%%%%%%%%%%%%%%%%%%%%%%%%%%%%%%
\begin{frame}{Bases for the Null Space of a Matrix}

\bbox
The \alert{null space} of an $m \times n$ matrix $A$, denoted \alert{$\Nul A$}, is the set of all solutions to the homogeneous equation $A \mathbf{x} = \mathbf{0}$.
\ebox

Let $\dsty A = \begin{bmatrix} 1 & 4 & 0 & 2 & -1\\ 3 & 12 & 1 & 5 & 5 \\ 2 & 8 & 1 & 3 & 2 \\ 5 & 20 & 2 & 8 & 8 \end{bmatrix}$.
\onslide<2->{
$\text{RREF}(A) = \begin{bmatrix} 1 & 4 & 0 & 2 & 0 \\ 0 & 0 & 1 & -1 & 0 \\ 0 & 0 & 0 & 0 & 1 \\ 0 & 0 & 0 & 0 & 0 \end{bmatrix}$.
}
\medskip

Find a basis for \alert{$\Nul A$}.

\onslide<2->{
\[ \mathbf{x} = \begin{bmatrix} -4x_2 - 2x_4 \\ x_2 \\ x_4 \\ x_4 \\ 0 \end{bmatrix} = x_2 \begin{bmatrix} -4 \\ 1 \\ 0 \\ 0 \\ 0 \end{bmatrix} + x_4 \begin{bmatrix} -2 \\ 0 \\ 1 \\ 1 \\ 0 \end{bmatrix} \quad 
\mathcal{B} = \left\{  \begin{bmatrix} -4 \\ 1 \\ 0 \\ 0 \\ 0 \end{bmatrix} , \begin{bmatrix} -2 \\ 0 \\ 1 \\ 1 \\ 0 \end{bmatrix} \right\} .\] 
}

\end{frame}

\begin{frame}{Basis for Column Space of a Matrix}

\vspace*{-.3em}
\bbox
The \alert{column space} of an $m \times n$ matrix $A = \begin{bmatrix} \mathbf{a}_1 &  \mathbf{a}_2 & \ldots &  \mathbf{a}_n  \end{bmatrix}$, denoted \alert{$\Col A$}, is the set of all linear combinations of the columns of $A$.
\ebox

\begin{columns}
  
\column{0.4\tw}
Find a basis for \alert{$\Col A$}.
\bigskip

\hspace*{1.5em} Let $\dsty A = 
\begin{bmatrix}
\onslide*<1-4>{
1 & 4 & 0 & 2 & -1\\
3 & 12 & 1 & 5 & 5 \\
2 & 8 & 1 & 3 & 2 \\ 
5 & 20 & 2 & 8 & 8}%
\onslide*<5->{
\colorr{1} & 4 & \colorr{0} & 2 & \colorr{-1}\\
\colorr{3} & 12 & \colorr{1} & 5 & \colorr{5} \\
\colorr{2} & 8 & \colorr{1} & 3 & \colorr{2} \\ 
\colorr{5} & 20 & \colorr{2} & 8 & \colorr{8}}%
\end{bmatrix}$.
\medskip

$\begin{array}{ll} R= \\ \text{RREF}(A) = \end{array} 
\begin{bmatrix} 
\onslide*<1-3>{
1 & 4 & 0 & 2 & 0 \\ 
0 & 0 & 1 & -1 & 0 \\ 
0 & 0 & 0 & 0 & 1 \\ 
0 & 0 & 0 & 0 & 0 }%
\onslide*<4->{
\colorr{1} & 4 & \colorr{0} & 2 & \colorr{0} \\ 
\colorr{0} & 0 & \colorr{1} & -1 & \colorr{0} \\ 
\colorr{0} & 0 & \colorr{0} & 0 & \colorr{1} \\ 
\colorr{0} & 0 & \colorr{0} & 0 & \colorr{0} }
\end{bmatrix}$.

\column{0.6\tw}

\pause
Note that $A\mathbf{y}=\mathbf{0} \ \colorb{\Longleftrightarrow} \ R\mathbf{y}=\mathbf{0}$.
\smallskip

\pause
So every linear dep of the columns of $R$ is 

a linear dep of the columns of $A$.
\smallskip

\pause
\alert{Pivot cols of $R$} are $\mathbf{e}_i$s; non-pivot cols are linear combs.
\smallskip

\pause
So \alert{pivot cols of $A$} are lin indep, form a basis for $\Col A$.
\medskip

\qquad $\dsty \mathcal{B} = \left\{ \begin{bmatrix} 1 \\ 3 \\ 2 \\ 5 \end{bmatrix} , \begin{bmatrix} 0 \\ 1 \\ 1 \\ 2 \end{bmatrix}  ,  \begin{bmatrix} -1 \\ 5 \\ 2 \\ 8 \end{bmatrix} \right\}$.

\end{columns}

\end{frame}

\begin{frame}{Basis for Column Space of a Matrix}
  \medskip
  
  \begin{columns}[T]
    \column{.3\textwidth}
      $A=
      \begin{bmatrix}
      \red{1} & 4 & \red{0} & 2 & \red{-1}\\
      \red{3} & 12 & \red{1} & 5 & \red{5} \\
      \red{2} & 8 & \red{1} & 3 & \red{2} \\ 
      \red{5} & 20 & \red{2} & 8 & \red{8}
      \end{bmatrix}$
      \medskip
    
    \column{.7\textwidth}
      Given $\b$ in $\Col A$, how can $\b$ be written as a linear combination of vectors in $\mathcal{B}$?
      \medskip
      
      \pause
      Solve $A\x=\b$, use the solution where the free variables are $0$.
      \vspace*{1.5em}
      
  \end{columns}
  
    \pause
    $\b=(1,12,3,17)$.
    %\medskip
    
    {\small
    $\begin{array}{l}\text{RREF of} \\ {[A|\b]}= \end{array}
    \begin{bmatrix}
      1 & 4 & 0 & 2 & 0 & 3\\
      0 & 0 & 1 & -1 & 0 & -7\\
      0 & 0 & 0 & 0 & 1 & 2\\
      0 & 0 & 0 & 0 & 0 & 0
    \end{bmatrix}$
    \qquad
    $\begin{bmatrix} x_1 \\ x_2 \\ x_3 \\ x_4 \\ x_5 \end{bmatrix} =
      \begin{bmatrix} 3 -4x_2 - 2x_4 \\ x_2 \\ -7+ x_4 \\ x_4 \\ 2 \end{bmatrix} = 
          \begin{bmatrix} 3 \\ 0  \\ -7 \\ 0 \\ 2 \end{bmatrix}
    + x_2 \begin{bmatrix} -4 \\ 1 \\ 0 \\ 0 \\ 0 \end{bmatrix}
    + x_4 \begin{bmatrix} -2 \\ 0 \\ 1 \\ 1 \\ 0 \end{bmatrix}$.
    \bigskip
    
    Set $x_2=x_4=0$.
    Then $\b=\begin{bmatrix} 1 \\ 12 \\ 3 \\ 17 \end{bmatrix} =
      3\begin{bmatrix} 
      \red{1} \\
      \red{3} \\
      \red{2} \\ 
      \red{5} 
      \end{bmatrix}
      -7
      \begin{bmatrix}
      \red{0}\\
      \red{1}\\
      \red{1}\\
      \red{2}
      \end{bmatrix}
      +2\begin{bmatrix}
      \red{-1}\\
      \red{5} \\
      \red{2} \\
      \red{8}
      \end{bmatrix}$.
    }  % small
    
\end{frame}


\begin{frame}{Basis for Row Space of a Matrix}

\vspace*{-.4em}
\bbox
\vspace*{-.2em}
The \alert{row space} of an $m \times n$ matrix $A = {\scriptsize \begin{bmatrix} \mathbf{r}_1 \\  \vdots \\  \mathbf{r}_m  \end{bmatrix}}$, denoted \alert{$\Row A$}, is the set of all linear combinations of the rows of $A$.
\quad $\Row A = \mbox{Span} \left\{ \mathbf{r}_1 , \mathbf{r}_2 , \ldots ,  \mathbf{r}_m \right\} = \Col A^T$
\ebox

\begin{columns}
  
\column{0.4\tw}
Find a basis for \alert{$\Row A$}.
\medskip

\hspace*{1.5em} Let $\dsty A = 
\begin{bmatrix}
1 & 4 & 0 & 2 & -1\\
3 & 12 & 1 & 5 & 5 \\
2 & 8 & 1 & 3 & 2 \\ 
5 & 20 & 2 & 8 & 8
\end{bmatrix}$.
\medskip

$\begin{array}{ll} R= \\ \text{RREF}(A) = \end{array} 
\begin{bmatrix} 
1 & 4 & 0 & 2 & 0 \\ 
0 & 0 & 1 & -1 & 0 \\ 
0 & 0 & 0 & 0 & 1 \\ 
0 & 0 & 0 & 0 & 0
\end{bmatrix}$.

\column{0.6\tw}
\pause
The rows of $R$ are linear combinations of rows of $A$.

\quad Thus, span rows $R$ \colorb{$\subseteq$} span rows $A$.

\pause
But since elementary row ops are \alert{reversible},

the rows of $A$ are linear combinations of rows of $R$.

Thus, span rows $A$ \colorb{$\subseteq$} span rows $R$.
\colorb{$\Rightarrow$} $\Row A = \Row R$.%
\medskip

\pause
So a basis for $\Row R$ is a basis for $\Row A$.
\medskip

A basis for $\Row A$ is 

\qquad $\left\{ ( 1, 4, 0, 2, 0), (0, 0, 1, -1, 0 ) ,(0, 0, 0, 0, 1 ) \right\}$.

\end{columns}

\end{frame}


\begin{frame}{Basis for Row Space of a Matrix}
  \bigskip

  \begin{theorem}
  If two matrices $A$ and $B$ are row equivalent, then their \alert{row spaces} are the same.
  \medskip

  If $B$ is in (reduced) row echelon form, the nonzero rows of $B$ form a basis for the row space of $A$ as well as for $B$.
  \end{theorem}
\end{frame}


\begin{frame}{Summary}
  \bigskip

  Given a matrix $A$:
  \medskip

  \bi
  \ii Solve the equation $A\mathbf{x} = \mathbf{0}$ to find a basis for \green{$\Nul A$}. \medskip
  \pause
  \ii Find the reduced row echelon form of $A$. \medskip
  \ii \quad The pivot columns of the \alert{original matrix $A$} form a basis for \alert{$\Col A$}. \medskip
  \ii \quad The non-zero rows of the \blue{RREF matrix} form a basis for \blue{$\Row A$}.
  \ei

\end{frame}

\end{document}
