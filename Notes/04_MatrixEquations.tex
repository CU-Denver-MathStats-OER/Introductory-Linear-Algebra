\documentclass[xcolor=dvipsnames,aspectratio=169,t]{beamer}
  % t means frames are vertically centered to the top
\usepackage{slides-header}
\title{Matrix Equations}

\begin{document}
\maketitle

\begin{frame}{Revisiting Linear Combinations}

  As we have previously seen, a very fundamental question in linear algebra is determining whether a vector $\mathbf{b}$ in $\mathbb{R}^n$ can be written as a linear combination of vectors $\mbf{v}_1, \mbf{v}_2, \ldots, \mbf{v}_p$ all in in $\mathbb{R}^n$.

  \begin{example}
    Determine whether  $\dsty \mbf{b} = \bbm 6 \\ -2 \\ -10 \ebm$  is in $ \mbox{Span} \left\{ \bbm 1 \\ 0 \\ -2  \ebm,  \bbm 0 \\ 1 \\ -1  \ebm \right\}$.
  \end{example}

    Do there exist real numbers (called \alert{weights}) \alert{$x_1$} and \alert{$x_2$} such that
    \[ \bbm 6 \\ -2 \\ -10 \ebm = \alert{x_1} \bbm 1 \\ 0 \\ -2  \ebm + \alert{x_2} \bbm 0 \\ 1 \\ -1  \ebm \mbox{?}\]

  
\end{frame}


\begin{frame}{Multiplying a Matrix and a Vector}

{\small
  \bbox
  Let $A$ be an $m \times n$ matrix with columns $\mathbf{a}_1, \mathbf{a}_2, \ldots , \mathbf{a}_n$ (each column in the matrix is in $\mathbb{R}^m$), and let $\mbf{x}$ denote a column vector in $\mathbb{R}^n$. Then we define the product of matrix $A$ and column vector $\mathbf{x}$ as
  \[ A\mbf{x} = \begin{bmatrix} \mbf{a}_1 & \mbf{a}_2 & \ldots & \mbf{a}_n \end{bmatrix} \begin{bmatrix} x_1 \\ x_2 \\ \vdots \\ x_n \end{bmatrix}
   = x_1  \mbf{a}_1 + x_2  \mbf{a}_2  + \ldots + x_n  \mbf{a}_n \]
   \ebox
}

{\small
   \begin{example}
    Do there exist real numbers (called \alert{weights}) \alert{$x_1$} and \alert{$x_2$} such that
    \[ \bbm 6 \\ -2 \\ -10 \ebm = \alert{x_1} \bbm 1 \\ 0 \\ -2  \ebm + \alert{x_2} \bbm 0 \\ 1 \\ -1  \ebm  = \begin{bmatrix} 1 & 0 \\ 0 & 1 \\ -2 & -1 \end{bmatrix} \alert{ \begin{bmatrix} x_1 \\ x_2 \end{bmatrix} } = A \alert{\mbf{x}} \mbox{?}\]
\end{example} }
   
  \end{frame}

\begin{frame}{Example}
  
  If possible, compute the product $A \mathbf{x}$. \ms
  
\begin{enumerate}
%\begin{multicols}{2}
\item $\dsty A = \begin{bmatrix} 2 & -4 & 3 & 1\\ 6 & 2 & 1 & 9\\ 1 & 0 & 2 &-1 \end{bmatrix}_{3\times 4}$ and $\dsty \mathbf{x} = \begin{bmatrix} -1 \\ -1 \\ 5 \\ 2 \end{bmatrix}_{4 \times 1}$
\vspace*{3em}

\item $\dsty A = \begin{bmatrix} 2 & -4 & 3 & 1\\ 6 & 2 & 1 & 9\\ 1 & 0 & 2 &-1 \end{bmatrix}_{3 \times 4}$ and $\dsty \mathbf{x} = \begin{bmatrix} -1 \\ -1 \\ 5  \end{bmatrix}_{3 \times 1}$
%\end{multicols}
\end{enumerate}

\vspace{2em}

\bbox
(\alert{!!}) $A \mbf{x}$ is only defined if the number of columns of $A$ equals the number of rows in $\mbf{x}$.
\ebox

\end{frame}

\begin{frame}

  \begin{columns}[T]

    \column{0.5\tw}
{\footnotesize
    \bbox
    Determine whether  $\dsty \mbf{b} = \bbm 6 \\ -2 \\ -10 \ebm$  is in $ \mbox{Span} \left\{ \bbm 1 \\ 0 \\ -2  \ebm,  \bbm 0 \\ 1 \\ -1  \ebm \right\}$.
    \ebox
    
We can set up the following matrix equation:  
 \[  \begin{bmatrix} 1 & 0 \\ 0 & 1 \\ -2 & -1 \end{bmatrix} \begin{bmatrix} x_1 \\ x_2 \end{bmatrix} =  \bbm 6 \\ -2 \\ -10 \ebm \]
 Which is equivalent to solving the system of linear equations:
 \[ \begin{array}{ccccc}
   x_1 &  &  & = & 6 \\
        &  & x_2 & = & -2 \\
   -2x_1 & + & -x_2 &=&-10
 \end{array}.
 \] }

 
 \column{0.5\tw}

 
 \bbox
 If $A$ is an $m \times n$ matrix with columns columns $\mathbf{a}_1, \mathbf{a}_2, \ldots , \mathbf{a}_n$ and if $\mbf{b}$ is in $\mathbb{R}^m$, the matrix equation $A\mbf{x} = \mbf{b}$ has the same solution set as the vector equation
   \[ x_1  \mbf{a}_1 + x_2  \mbf{a}_2  + \ldots + x_n  \mbf{a}_n = \mbf{b} \]
   which has corresponding augmented matrix
   \[ \begin{bmatrix} \mbf{a}_1 & \mbf{a}_2 & \ldots & \mbf{a}_n & \mbf{b} \end{bmatrix} \]
   \ebox
   
\end{columns}
 
 \end{frame}

\begin{frame}

\bbox
 If $A$ is an $m \times n$ matrix with columns vectors $\mathbf{a}_1, \mathbf{a}_2, \ldots , \mathbf{a}_n$ and if $\mbf{b}$ is in $\mathbb{R}^m$, the matrix equation $A\mbf{x} = \mbf{b}$ has the \alert{same solution set} as the vector equation
   \[ x_1  \mbf{a}_1 + x_2  \mbf{a}_2  + \ldots + x_n  \mbf{a}_n = \mbf{b}, \]
   which has corresponding augmented matrix
   \[ \begin{bmatrix} \mbf{a}_1 & \mbf{a}_2 & \ldots & \mbf{a}_n & \mbf{b} \end{bmatrix}. \]
   \ebox
  
  \begin{theorem}
    The equation $A \mathbf{x} = \mbf{b}$ has a solution (is consistent) if and only if $\mathbf{b}$ is a linear combination of the columns of matrix $A$.
  \end{theorem}
  
\end{frame}

\begin{frame}

  \bi
  \ii We have previously considered whether a specified  vector is in  $\mbox{Span} \left\{ \mbf{v}_1,  \ldots , \mbf{v}_p \right\}$.
  \ii A more abstract question is whether \alert{all vectors} are in the $\mbox{Span} \left\{ \mbf{v}_1,  \ldots , \mbf{v}_p \right\}$.
  \ei

  \bbox
  Are all vectors $\dsty \mbf{b} = \bbm b_1 \\ b_2 \\ b_3 \ebm$ in $\mathbb{R}^3$ in $\dsty \mbox{Span} \left\{ \bbm 1\\ 1\\ -3 \ebm ,  \bbm 1 \\ 0 \\ 2 \ebm , \bbm 6 \\ 8 \\ 6 \ebm \right\}$?
  \ebox

  \ms
  
  \[ \bbm 1 & 1 & 6 & b_1 \\ 1 & 0 & 8 & b_2 \\ -3 & -3 & 6 & b_3 \ebm
\rightarrow  \bbm \alert{1} & 1 & 6 & b_1 \\ 0 & \alert{-1} & 2 & b_2-b_1 \\ -3 & -3 & 6 & b_3 \ebm
  \rightarrow  \bbm \alert{1} & 1 & 6 & b_1 \\ 0 & \alert{-1} & 2 & b_2-b_1 \\ 0 & 0 & \alert{24} & b_3+3b_1 \ebm   \]

  \bbox
  No matter the values of $b_1$, $b_2$, and $b_3$, we see that  $x_1$, $x_2$, and $x_3$ are all \alert{basic variables}. The system is consistent, which means all vectors are in the span.
  \ebox
  
\end{frame}

\begin{frame}
  \bbox
  Let $A$ be an $m \times n$ matrix. Then the following statements are all equivalent.
  \bb
  \ii For any vector $\mbf{b}$ in $\mathbb{R}^m$, the equation $A\mbf{x} = \mbf{b}$ has a solution.
  \ii All vectors $\mbf{b}$ in $\mathbb{R}^m$ can be written as a linear combination of the columns of $A$.
  \ii The columns of $A$ span all of $\mathbb{R}^m$.
  \ii When forming the RREF of $A$, there is a pivot position in every row.
  \ee
  \ebox
\end{frame}

\begin{frame}{Example}
  \bbox
  Do the vectors $\dsty 
    \bbm 1 \\ 3 \\ 6 \\ -2 \ebm,
    \bbm 0\\ 1 \\ 2 \\ 0 \ebm,
    \bbm 1 \\ 4 \\ 9 \\ -4 \ebm , 
    \mbox{ and } \bbm 2 \\ 2 \\ 2 \\ 0 \ebm$ span all of $\mathbb{R}^4$?
  \ebox
  \bigskip

  \pause
  {\small
  \[ \hspace*{-1em}
  \bbm 1 & 0 & 1 & 2 \\
  3 & 1 & 4 & 2 \\
  6 & 2 & 9 & 2\\
  -2 & 0 & -4 & 0 \ebm \rightarrow
  \bbm 1 & 0 & 1 & 2 \\
  0 & 1 & 1 & -4 \\
  0 & 2 & 3 & -10\\
  0 & 0 & -2 & 4 \ebm \rightarrow
  \bbm 1 & 0 & 1 & 2 \\
  0 & 1 & 1 & -4 \\
  0 & 0 & 1 & -2\\
  0 & 0 & -2 & 4 \ebm \rightarrow
  \bbm 1 & 0 & 1 & 2 \\
  0 & 1 & 1 & -4 \\
  0 & 0 & 1 & -2\\
  0 & 0 & 0 & 0 \ebm 
  \rightarrow
  \bbm 1 & 0 & 0 & 4 \\
  0 & 1 & 3 & -2 \\
  0 & 0 & 1 & -2\\
  \alert{0} & \alert{0} & \alert{0} & \alert{0} \ebm \]
  }
  \bigskip

  {\small 
  Since the matrix $A$ does \alert{NOT} have a pivot position in every row, the vectors do \alert{NOT} span all of $\mathbb{R}^4$.}

\end{frame}


\begin{frame}{Properties of Matrix-Vector Products}

 { \small
  \bbox
  Let $A$ be an $m \times n$ matrix, $\mbf{u}$ and $\mbf{v}$ be vectors in $\mathbb{R}^n$, and $c$ be a scalar. We have
  \bi
  \ii $\dsty A(\mathbf{u} + \mbf{v}) = A\mbf{u} +A \mbf{v}$
  \ii $\dsty A(c \mbf{v}) = c \left(A \mbf{v} \right)$
  \ei
  \ebox

  \begin{proof}
    \pause
    Let $\mbf{a}_1, \mbf{a}_2, \ldots, \mbf{a}_n$ denote the column vectors of $A$. We have
    \begin{align*}
      A(\mathbf{u} + \mbf{v}) 
      = \bbm \mbf{a}_1 & \mbf{a}_2 & \ldots & \mbf{a}_n \ebm \bbm u_1+v_1 \\ u_2+v_2 \\ \vdots \\ u_n+v_n \ebm
      &= (u_1+v_1) \mbf{a}_1 + (u_2+v_2) \mbf{a}_2 + \ldots + (u_n+v_n) \mbf{a}_n \\
      &= \left( u_1 \mbf{a}_1 + u_2 \mbf{a}_2 + \ldots + u_n\mbf{a}_n \right) +  \left( v_1 \mbf{a}_1 + v_2 \mbf{a}_2 + \ldots + v_n\mbf{a}_n \right) \\
      &= A \mbf{u} + A \mbf{v}.\qedhere
    \end{align*}
  \end{proof}
 }
\end{frame}

\end{document}
