\documentclass[xcolor=dvipsnames,aspectratio=169,t]{beamer}
  % t means frames are vertically centered to the top
\usepackage{slides-header}
\title{Homogeneous Systems of Linear Equations}

\begin{document}
\maketitle

\begin{frame}{Homogeneous Solutions}
  \bbox
  A system of equations is \alert{homogeneous} if it can be written in the form $A \mathbf{x} = \mathbf{0}$.
  \ebox
  \vspace*{-1.5em}

  \begin{columns}
    \column{0.5\tw}
    \[
    \begin{array}{l}
     x_1 +x_2 +x_3=7 \\
     x_1 -x_2 +2x_3= 7\\
     5x_1+x_2+x_3=11
    \end{array}  \]

    \[   \bbm
    1 & 1 & 1 \\
    1 & -1 & 2 \\
    5 & 1 & 1 \ebm \bbm x_1 \\ x_2 \\ x_3 \ebm = \bbm 7 \\ 7 \\ 11\ebm  \]

    \hspace*{2em} \alert{NOT a homogeneous system.}

    \column{0.5\tw}
    \[
    \begin{array}{l}
     x_1 +x_2 +x_3=0 \\
     x_1 -x_2 +2x_3= 0\\
     5x_1+x_2+x_3=0
    \end{array}  \]

    \[   \bbm
    1 & 1 & 1 \\
    1 & -1 & 2 \\
    5 & 1 & 1 \ebm \bbm x_1 \\ x_2 \\ x_3 \ebm = \bbm 0 \\ 0 \\ 0\ebm  \]

    \hspace*{2em} \colorb{YES, this is a homogeneous system.}
  \end{columns}
  \medskip

  \pause
  \bbox
    Every homogeneous system has a \alert{trivial solution} $\x=\mathbf{0}$, and so is \alert{consistent}.
    \smallskip
    
    Does the homogeneous system have a \colorb{nontrivial solution}?
    ie, infinite number of solutions?
  \ebox

\end{frame}

\begin{frame}{Existence of Nontrivial Solutions}

  Recall that if the system $A \mathbf{x} = \mathbf{b}$ is \alert{consistent}, then it has:
  \begin{itemize}
  \item A \alert{unique solution} if, when forming the RREF of $A$, there is a pivot position in every column (ie, all variables are \colorb{basic}).
  \item An \colorb{infinite number} of solutions if there is a column of $A$ without a pivot position (ie, there is at least one \colorb{free variable}).
  \end{itemize}
  \bigskip

  \begin{theorem}
  The homogeneous equation $A\mathbf{x} = \mathbf{0}$ has a nontrivial solution if and only if the solution set has at least one free variable.
  \end{theorem}

\end{frame}

\begin{frame}{Example}
\medskip
   
    \begin{columns}
      \column{0.6\tw}
     Find the solution set to the \alert{homogeneous} system
    \[   \begin{array}{ccccccc}
     x_1 & + & 3x_2 & + &7 x_3&=&0\\
     &  & x_2 & + & 3x_3 &=& 0\\
    -2x_1 & - & 4x_2 & - & 8x_3 &=& 0
    \end{array}  \]

    \column{0.4\tw}
    \end{columns}
  
  \bigskip
    
\[ \bbm 1 & 3 & 7 & 0\\
0 & 1 & 3 & 0\\
-2 & -4 & -8 & 0 \ebm \sim
\bbm 1 & 3 & 7 & 0\\
0 & 1 & 3 & 0\\
0 & 2 & 6 & 0 \ebm \sim
\bbm 1 & 3 & 7 & 0\\
0 & 1 & 3 & 0\\
0 & 0 & 0 & 0 \ebm \sim
\bbm \alert{1} & 0 & \colorb{-2} & 0\\
0 & \alert{1} & \colorb{3} & 0\\
0 & 0 & \colorb{0} & 0 \ebm \]
\smallskip

Notice \colorb{$x_3$ is a free variable}, and the solution set can therefore be given by

\[ \mathbf{x} = \bbm x_1 \\ x_2 \\ x_3 \ebm = \bbm 2x_3 \\ -3x_3 \\ x_3 \ebm =  x_3 \bbm 2 \\ -3 \\ 1 \ebm = x_3 \mbf{v} \ , \mbox{ where } \mbf{v} = \bbm 2 \\ -3 \\ 1 \ebm . \]
\end{frame}


\begin{frame}{Example}
\medskip

  Find the solution set to the homogeneous system
  \[   \begin{array}{ccccccc}
     3x_1 & - & 2x_2 & + & x_3&=&0\\
     -6x_1& + & 4x_2 & - & 2x_3 &=& 0
  \end{array}  \]

  \[ \bbm 3 & -2 & 1 & 0\\
  -6 & 4 & -2 & 0 \ebm \sim
  \bbm 3 & -2 & 1 & 0\\
  0 & 0 & 0 & 0 \ebm \sim
  \bbm \alert{1} & \colorb{\frac{-2}{3}} & \colorb{\frac{1}{3}} & 0\\
  0 & \colorb{0} & \colorb{0} & 0  \ebm \]
  \medskip

  Both \colorb{$x_2$ and $x_3$ are free variables}, and the solution set can therefore be given by
  \[ \mathbf{x} = \bbm x_1 \\ x_2 \\ x_3 \ebm = \bbm \frac{2}{3}x_2-\frac{1}{3} x_3 \\ x_2 \\ x_3 \ebm =  x_2 \bbm \frac{2}{3} \\ 1 \\ 0 \ebm +  x_3 \bbm -\frac{1}{3} \\ 0 \\ 1 \ebm = x_2 \mbf{u}  + x_3 \mbf{v}. \]
\end{frame}


% \begin{frame}{Parametric Vector Form}
% 
%       Find the solution set to the homogeneous system
% \[   \begin{array}{ccccccc}
%      3x_1 & - & 2x_2 & + & x_3&=&0\\
%      -6x_1& + & 4x_2 & - & 2x_3 &=& 0
% \end{array}  \]
% 
% \bi
% \ii \alert{Implicit Solution:} The plane $\dsty 3x_1 - 2x_2 + x_3=0$.
% \ii \alert{Explicit Solution:} The plane $\dsty x_1 = \frac{2}{3}x_2-\frac{1}{3} x_3$.
% \ii \alert{Vector Form:} $\dsty x_2 \bbm \frac{2}{3} \\ 1 \\ 0 \ebm +  x_3 \bbm -\frac{1}{3} \\ 0 \\ 1 \ebm = x_2 \mbf{u}  + x_3 \mbf{v}$
% \ii \alert{Parametric Vector Form:} $\dsty \colorb{s} \bbm 2 \\ 3 \\ 0 \ebm +  \colorb{t} \bbm -1 \\ 0 \\ 3 \ebm = \colorb{s} \mbf{u}  + \colorb{t} \mbf{v}$ with $\colorb{s,t}$ in $\mathbb{R}$
% \medskip
% 
% The weights $\colorb{s,t}$ are any scalars. Vector Form is a special case of Parametric Vector Form.
% \ei
% 
% \end{frame}

\begin{frame}{Nonhomogeneous Linear Systems}
\medskip

  Find the solution set to the \alert{nonhomogeneous} system
  \[
  \begin{array}{ccccccc}
    x_1 & + & 3x_2 & + &7 x_3&=&4\\
    &  & x_2 & + & 3x_3 &=& 5\\
  -2x_1 & - & 4x_2 & - & 8x_3 &=& 2
  \end{array}  \]
  
  \pause
  \[ \bbm 1 & 3 & 7 & 4\\
  0 & 1 & 3 & 5\\
  -2 & -4 & -8 & 2 \ebm \sim
  \bbm 1 & 3 & 7 & 4\\
  0 & 1 & 3 & 5\\
  0 & 2 & 6 & 10 \ebm \sim
  \bbm 1 & 3 & 7 & 4\\
  0 & 1 & 3 & 5\\
  0 & 0 & 0 & 0 \ebm \sim
  \bbm \alert{1} & 0 & \colorb{-2} & -11\\
  0 & \alert{1} & \colorb{3} & 5\\
  0 & 0 & \colorb{0} & 0 \ebm \]
  \medskip

  Notice \colorb{$x_3$ is a free variable}, and the solution set can therefore be given by

  \[ \mathbf{x} = \bbm x_1 \\ x_2 \\ x_3 \ebm = \bbm -11+2x_3 \\ 5-3x_3 \\ x_3 \ebm =  \bbm -11 \\ 5 \\ 0 \ebm +  x_3 \bbm 2 \\ -3 \\ 1 \ebm = \mathbf{p} + t \mbf{v} \ \ (t \mbox{ in } \mathbb{R}) . \]
\end{frame}


\begin{frame}{Comparing Homogeneous and Nonhomogeneous Solutions}
  \vspace*{-2em}
  
  \begin{columns}
\column{0.45\tw}

\[   \begin{array}{ccccccc}
     x_1 & + & 3x_2 & + &7 x_3&=&0\\
     &  & x_2 & + & 3x_3 &=& 0\\
    -2x_1 & - & 4x_2 & - & 8x_3 &=& 0
\end{array}  \]

\[ \mathbf{x} =  x_3 \bbm 2 \\ -3 \\ 1 \ebm = t \mbf{v} \ \ (t \mbox{ in } \mathbb{R}) . \]

    \column{0.55\tw}  % why are the columns not vertically aligned since they have the same number of lines?

\[   \begin{array}{ccccccc}
     x_1 & + & 3x_2 & + &7 x_3&=&4\\
     &  & x_2 & + & 3x_3 &=& 5\\
    -2x_1 & - & 4x_2 & - & 8x_3 &=& 2
\end{array}  \]

\[ \mathbf{x} =  \alert{\bbm -11 \\ 5 \\ 0 \ebm} +  x_3 \bbm 2 \\ -3 \\ 1 \ebm =\alert{\mathbf{p}} + t \mbf{v} \ \ (t \mbox{ in } \mathbb{R}) . \]%
  \end{columns}

  \begin{center}
    \begin{tikzpicture}
      \draw (-1,0)--(5,0);
      \draw (0,-1)--(0,3);
      \draw[very thick,blue] (-1,-.5)--(4,2);
      \node[anchor=west] at (4,2) {$A\x=0$};  % has slope .5
      \draw[very thick,red] (-1,.5)--(4,3);
      \node[anchor=west] at (4,3) {$A\x=\b$};  % has equation y=.5*x+1
      
      \coordinate (p) at (1,1.5);
      \node[anchor=south] at (p) {$\mathbf{p}$};
      \draw[very thick,red,-latex] (0,0)--(p);
      \node[circle,fill,inner sep=1.5pt] at (p) {};
      
      \coordinate (v) at (1.5,.75);
      \node[anchor=north] at (v) {$\v$};
      \node[circle,fill,inner sep=1.5pt] at (v) {};
      
      \coordinate (tv) at (2.25,1.125);
      \node[anchor=north] at (tv) {$t\v$};
      \node[circle,fill,inner sep=1.5pt] at (tv) {};
      
      \coordinate (pplustv) at (3.25,2.625);
      \node[anchor=south east] at (pplustv) {$\mathbf{p}+t\v$};
      \node[circle,fill,inner sep=1.5pt] at (pplustv) {};
      \draw[very thick,black,-latex] (tv)--(pplustv);
    \end{tikzpicture}
  \end{center}
\end{frame}


\begin{frame}

  \begin{theorem}
    Suppose \colorb{$A \mathbf{x} = \mbf{b}$} is consistent and has a solution \colorb{$\mathbf{x} = \mathbf{p}$}.
    If $\colorr{\mbf{v}}$ is \textbf{any} solution to the \colorr{homogeneous} equation \colorr{$A \mathbf{x} = \mathbf{0}$}, then
    \[ \mbf{x} = \colorb{\mbf{p}} + \colorr{\mbf{v}} \]
    is a solution to the \colorb{nonhomogeneous} equation \colorb{$A \mathbf{x} = \mathbf{b}$}.
  \end{theorem}

  \begin{proof}
    \pause
    We have $A \mathbf{x} = A \left( \colorb{ \mbf{p}} + \alert{\mbf{v}} \right) = \colorb{A  \mbf{p}} + \alert{A  \mbf{v}} = \colorb{\mbf{b}} + \alert{\mathbf{0}} = \colorb{\mathbf{b}}$.
  \end{proof}
  
  \pause
  
  \begin{theorem}
    \textbf{Every} solution $\mathbf{x}$ of $\colorb{A \mathbf{x} = \mbf{b}}$ can be written as $\mbf{x} = \colorb{\mbf{p}} + \colorr{\mbf{v}}$ for \textbf{some} solution $\colorr{\mbf{v}}$ of \colorr{$A \mathbf{x} = \mathbf{0}$}.
  \end{theorem}

  \begin{proof}
    \pause
    Let $\mathbf{y}$ be a solution to $\colorb{A \mathbf{x} = \mathbf{b}}$.
    Let $\colorr{\mbf{v}}=\mathbf{y}-\colorb{\mathbf{p}}$.
    Then $A\colorr{\mbf{v}}=A(\mathbf{y}-\colorb{\mathbf{p}}) = A\mathbf{y}-A\colorb{\mathbf{p}}= \colorb{\mathbf{b}}-\colorb{\mathbf{b}}=\colorr{\mathbf{0}}$.
    Hence, $\colorr{\mbf{v}}$ is a solution to $\colorr{A \mathbf{x} = \mathbf{0}}$.
  \end{proof}
  
  \end{frame}

  \end{document}
