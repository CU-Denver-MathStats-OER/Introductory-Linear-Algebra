\documentclass[xcolor=dvipsnames,aspectratio=169,t]{beamer}
  % t means frames are vertically centered to the top
\usepackage{slides-header}
\title{Orthogonal Sets}

\begin{document}
\maketitle

\begin{frame}{Orthogonal Sets}
  \begin{definition}
    A set of vectors $\left\{ \v_1, \v_2, \ldots , \v_p \right\}$ in $\R^n$ is said to be an \alert{orthogonal set} 

    \quad if $\v_i \cdot \v_j =0$ when $i \ne j$ (ie, the vectors are pairwise orthogonal).
  \end{definition}
  \medskip

  \begin{example}
  Is $\left\{ \begin{bmatrix} 1 \\ -1 \\ 0 \end{bmatrix}, \begin{bmatrix} 1 \\ 1 \\ 0 \end{bmatrix}, \begin{bmatrix} 0 \\ 0 \\ 1 \end{bmatrix} \right\}$ an orthogonal set?
  \end{example}
  \medskip

  \pause
  Since
  \[ \v_1 \cdot \v_2 = 0 , \quad \v_1 \cdot \v_3 = 0, \quad \mbox{and } \v_2 \cdot \v_3 = 0 ,\]
  the set is indeed an orthogonal set.
\end{frame}


\begin{frame}{Linear Independence of Orthogonal Sets}
  \begin{theorem}
  If $S = \left\{ \v_1 , \v_2 , \ldots , \v_p \right\}$ is an \alert{orthogonal set} of \alert{nonzero} vectors in $\R^n$, then
  \smallskip
  
  $S$ is a \alert{linearly independent} set, and is therefore a \blue{basis} for the subspace spanned by $S$.
  \end{theorem}
  \medskip

  \pause
  \blue{Proof.}\smallskip
  
  Suppose $S$ is an orthogonal set and that the vectors in $S$ are linearly \alert{dependent}.
  This implies there exist scalars $c_1,c_2,\ldots,c_p$, with at least one $c_i \ne 0$,
  such that $\mathbf{0} = c_1 \v_1 + c_2 \v_2 + \ldots + c_p \v_p$.
  \pause Then we have

  \begin{columns}[T]
  \column{0.5\tw} 
  \vspace*{-1em}
  \begin{align*}
  0 &= \mathbf{0} \cdot \v_1 = (c_1 \v_1 + c_2 \v_2 + \ldots + c_p \v_p) \cdot \v_1 \\
  &= c_1( \v_1 \cdot \v_1) + c_2 ( \v_2 \cdot \v_1) + \ldots c_p ( \v_p \cdot \v_1) \\
  &= c_1( \v_1 \cdot \v_1).
  \end{align*}

  \column{0.5\tw}
  \pause
  Since $\v_1$ is nonzero, we have $c_1=0$.
  Similarly, we can show that $c_1 = c_2 = \ldots = c_p =0$,
  which contradicts our original assumption.
  Thus $S$ is \alert{linearly independent}.
  \end{columns}
  
  \hfill\blue{$\qed$}
\end{frame}


\begin{frame}{Orthogonal Bases}
  \medskip
  
  \begin{definition}
    An \alert{orthogonal basis} for a subspace $W$ of $\R^n$ is a \blue{basis} for $W$ that is also an \blue{orthogonal set}.
  \end{definition}
  \medskip

  \begin{example}
  \bi
  \ii The set $S_1  = \left\{ \begin{bmatrix} 1 \\ -1 \\ 0 \end{bmatrix}, \begin{bmatrix} 1 \\ 1 \\ 0 \end{bmatrix}, \begin{bmatrix} 0 \\ 0 \\ 1 \end{bmatrix} \right\}$ is an orthogonal basis for $\R^3$. \ms
  \ii The set $S_2  = \left\{ \begin{bmatrix} 1 \\ -1 \\ 1 \end{bmatrix}, \begin{bmatrix} 1 \\ 1 \\ 0 \end{bmatrix}, \begin{bmatrix} 0 \\ 0 \\ 1 \end{bmatrix} \right\}$ is a basis for $\R^3$ that is \alert{not} orthogonal.
  \ei
  \end{example}
\end{frame}


\begin{frame}{Finding Coordinates With an Orthogonal Basis}
  \bigskip

  Suppose that $\B = \left\{ \u_1 , \u_2, \ldots , \u_p \right\}$ is an \alert{orthogonal basis} for a subspace $W$ of $\R^n$.
  \smallskip
  
  Find $\B$-coordinates of $\y$ in $W$.

  \pause
  \begin{columns}[T]
  \column{0.45\tw}
  \begin{align*}
    \onslide<+-> { \y &= c_1 \u_1 + c_2 \u_2 + \ldots + c_p \u_p \\}
    \onslide<+-> { \y \cdot \u_1 &= (c_1 \u_1 + c_2 \u_2 + \ldots + c_p \u_p) \cdot \u_1 \\}
    \onslide<+-> { \y \cdot \u_1 &= c_1 (\u_1 \cdot \u_1) + \ldots + c_p(\u_p \cdot \u_1) \\
                   \y \cdot \u_1 &= c_1 (\u_1 \cdot \u_1)\\[.3em]
                   c_1 &= \frac{\y \cdot \u_1}{\u_1 \cdot \u_1} \\[.7em]}
    \onslide<+-> { c_j &= \frac{\y \cdot \u_j}{\u_j \cdot \u_j} \text{ for each $j$.} }
  \end{align*}

  \column{0.55\tw}
  \pause
  \begin{theorem}
  Let $S= \left\{ \u_1 , \u_2, \ldots , \u_p \right\}$ be an orthogonal basis for a subspace $W$ of $\R^n$. For each $\y$ in $W$, the weights in the linear combination $\y = c_1 \u_1 + c_2 \u_2 + \ldots + c_p \u_p$ are given by
  \[ c_j =  \frac{\y \cdot \u_j}{\u_j \cdot \u_j}  \quad \mbox{for } j = 1, 2, \ldots , p .\]
  \end{theorem}
  \end{columns}
\end{frame}


\begin{frame}{Finding Coordinates with an Orthogonal Basis}
  \begin{columns}
  \column{0.6\tw}
  \begin{example}
  Express the vector $\y = \begin{bmatrix} 3 \\ 7 \\ 4 \end{bmatrix}$ as a linear combination of the orthogonal basis $\left\{ \begin{bmatrix} 1 \\ -1 \\ 0 \end{bmatrix}, \begin{bmatrix} 1 \\ 1 \\ 0 \end{bmatrix}, \begin{bmatrix} 0 \\ 0 \\ 1 \end{bmatrix} \right\}$.
  \end{example}

  \pause
  \column{0.4\tw}
  \[ \alert{c_1 = \frac{\y \cdot \u_1}{\u_1 \cdot \u_1} = \frac{3-7}{2} = -2} \]

  \[ \colorb{c_2 = \frac{\y \cdot \u_2}{\u_2 \cdot \u_2} = \frac{3+7}{2} = 5} \]

  \[ \colorg{c_3 = \frac{\y \cdot \u_3}{\u_3 \cdot \u_3} = \frac{4}{1} = 4} \]
  \end{columns}

  \ms

  Thus we have 
  
  \[ \y = \begin{bmatrix} 3 \\ 7 \\ 4 \end{bmatrix}
  = \alert{-2}  \begin{bmatrix} 1 \\ -1 \\ 0 \end{bmatrix} 
  + \colorb{5}  \begin{bmatrix} 1 \\ 1 \\ 0 \end{bmatrix} 
  + \colorg{4} \begin{bmatrix} 0 \\ 0 \\ 1 \end{bmatrix} 
  .\]
\end{frame}


\begin{frame}{Orthogonal Projections}
  \bigskip

  \bi
  \ii In physics, forces are generally expressed as vectors, $\mathbf{F}$.
  \ii An object is moving with velocity $\v$.
  \ii We often want to decompose the force $\mathbf{F}$ into two components $\mathbf{F} = \mathbf{F}_{\rm perp} + \mathbf{F}_{\rm parallel}$
  %\ii In data science, we may want to minimize the sum of the squared distances of observed values from a plane of predicted values.
  \ei
  \vspace*{1.5em}

  \begin{columns}[T]

  \column{0.7\tw}

  Given a nonzero vector $\u$ in $\R^n$, we wish to decompose a vector $\y$ in $\R^n$ into the sum of two vectors, one that is \alert{parallel} to $\u$ and the other is \alert{orthogonal} to $\u$:
  \[ \y = \hat{\y} + \z = c \u + \z =  \mbox{proj}_{\u} \y + \y_{\rm perp} = \mbox{proj}_{\u} \y + (\y - \hat{\y}) .\]

  \column{0.3\tw}
  \hspace*{-1em}
  
% picture of orthogonal projection onto subspace W

  \begin{tikzpicture}
    \draw[fill=black!5] (-1.5,-.25)--(.25,.75)--(2.25,-.25)--(.5,-1.25)--cycle;
    \node[anchor=north] at (-1,-.5) {$W$};
  
    \coordinate (origin) at (0,0);
    \node[anchor=north east] at (origin) {$0$};
    
    \coordinate (y) at (1,1);
    \draw[very thick,black] (origin)--(y);
    \node[anchor=west] at (y) {$\y$};
    \node[circle,fill,inner sep=1.5pt] at (y) {};
    
    \coordinate (yhat) at (1,-.5);
    \draw[very thick,red] (origin)--(yhat);
    \draw[dashed] (y)--(yhat);
    \draw[thick,black] (.8,-.40)--(.8,-.20)--(1,-.30);
    \node[anchor=west] at (yhat) {$\hat\y=\proj_W \y$};
    \node[circle,fill,inner sep=1.5pt] at (yhat) {};
    
    \coordinate (z) at (0,1.5);
    \draw[very thick,blue] (origin)--(z);
    \node[anchor=south west] at (z) {$\z=\y-\hat\y$};
    \node[circle,fill,inner sep=1.5pt] at (z) {};
  \end{tikzpicture}

  \end{columns}
\end{frame}


\begin{frame}
  \vspace*{-.7em}

  \begin{columns}[T]
  \column{0.7\tw}
  Given a nonzero vector $\u$ in $\R^n$, we wish to decompose a vector $\y$ in $\R^n$ into the sum of two vectors, one is parallel to $\u$ and the other is orthogonal to $\u$:
  \vspace*{-2em}
  
  \[ \y = \hat{\y} + \z = c \u + \z =  \mbox{proj}_{\u} \y + \y_{\rm perp} = \mbox{proj}_{\u} \y + (\y - \hat{\y}) .\]

  \column{0.3\tw}
  \hspace*{2em}\scalebox{.6}{
% picture of orthogonal projection onto subspace W

  \begin{tikzpicture}
    \draw[fill=black!5] (-1.5,-.25)--(.25,.75)--(2.25,-.25)--(.5,-1.25)--cycle;
    \node[anchor=north] at (-1,-.5) {$W$};
  
    \coordinate (origin) at (0,0);
    \node[anchor=north east] at (origin) {$0$};
    
    \coordinate (y) at (1,1);
    \draw[very thick,black] (origin)--(y);
    \node[anchor=west] at (y) {$\y$};
    \node[circle,fill,inner sep=1.5pt] at (y) {};
    
    \coordinate (yhat) at (1,-.5);
    \draw[very thick,red] (origin)--(yhat);
    \draw[dashed] (y)--(yhat);
    \draw[thick,black] (.8,-.40)--(.8,-.20)--(1,-.30);
    \node[anchor=west] at (yhat) {$\hat\y=\proj_W \y$};
    \node[circle,fill,inner sep=1.5pt] at (yhat) {};
    
    \coordinate (z) at (0,1.5);
    \draw[very thick,blue] (origin)--(z);
    \node[anchor=south west] at (z) {$\z=\y-\hat\y$};
    \node[circle,fill,inner sep=1.5pt] at (z) {};
  \end{tikzpicture}
}
  \end{columns}
  \medskip

  We want to find $\hat{\y} = \mbox{proj}_{\u} \y= c\u$. 
  \pause
  The other component $\z = \y - \hat{\y}$ is \alert{orthogonal} to $\u$, so
  \[ 0 = (\y - c\u) \cdot \u = (\y \cdot \u) - c( \u \cdot \u) .\]
  \pause
  Thus we see that 
  \[ c = \frac{\y \cdot \u}{\u \cdot \u}. \]

  \pause
  \begin{definition}
    The vector $\hat{\y}$ is called the \alert{orthogonal projection of $\y$ onto $\u$}, and the vector $\z$ is called the \alert{component of $\y$ orthogonal to $\u$}. We have 
    \[ \hat{\y} =  \mbox{proj}_{\u} \y =  \left( \frac{\y \cdot \u}{\u \cdot \u} \right) \u, \qquad \z = \y - \hat{\y}.\]
  \end{definition}
\end{frame}


\begin{frame}{Application of Projections}

\begin{example}
Suppose a sailboat has constant velocity given by $\v = \begin{bmatrix} 3 \\ 4 \end{bmatrix}$. The wind has a constant velocity of $\w = \begin{bmatrix} 10 \\ -10 \end{bmatrix}$. Find the component of the force that in the direction of the sailboat's motion.
\end{example} 

\pause
\begin{columns}[T]

\column{0.5\tw}

We have 

\[ \hat{\w} =  \mbox{proj}_{\v} \w =  \left( \frac{\w \cdot \v}{\v \cdot \v} \right) \v = \frac{-10}{25} \v =\begin{bmatrix}  -6/5 \\ -8/5 \end{bmatrix} \]

\[ \z = \w - \hat{\w} = \begin{bmatrix} 56/5 \\  -42/5 \end{bmatrix} \]

\column{0.5\tw}

And we can verify that

\[ \hat{\w} \cdot \z = \begin{bmatrix}  -6/5 \\ -8/5 \end{bmatrix} \cdot \begin{bmatrix} 56/5 \\  -42/5 \end{bmatrix} =  0 .\]

\end{columns}
\end{frame}


\begin{frame}{Orthonormal Sets}
  \medskip

  \begin{definition}
  A set $\left\{ \u_1, \u_2, \ldots , \u_p \right\}$ is an \alert{orthonormal set} if it is an orthogonal set of \alert{unit} vectors.
  \end{definition}
  \medskip

  \begin{example}
    The set $S = \left\{ \e_1,\e_2,\ldots,\e_n \right\}$ is an orthonormal set that forms a basis for $\R^n$.
  \end{example}
  \medskip

  \bi
  \ii Orthonormal sets (bases) are particularly nice to work with as many formulas simplify considerably.
  \ii For example, $\dsty \hat{\y} =  \mbox{proj}_{\u} \y =  \left( \frac{\y \cdot \u}{\u \cdot \u} \right) \u = (\y \cdot \u) \u$.
  \ei
\end{frame}

\begin{frame}{Matrices Constructed from Orthonormal Sets}
  \bigskip

  Let $\left\{ \u_1, \u_2, \ldots, \u_p \right\}$ be an \alert{orthonormal set} in $\R^n$ and let $U = \begin{bmatrix} \u_1 & \u_2 & \ldots & \u_p \end{bmatrix}_{n\times p}$. Then

  {\small
  \[ U^TU = \begin{bmatrix} \u_1^T \\ \u_2^T \\ \vdots \\ \u_p^T \end{bmatrix}_{p \times n} \begin{bmatrix} \u_1 & \u_2 & \ldots & \u_p \end{bmatrix}_{n \times p} 
  \pause =
  \begin{bmatrix} \u_1^T  \u_1 & \u_1^T \u_2 & \ldots &  \u_1^T \u_p \\[.4em]
  \u_2^T \u_1 & \u_2^T \u_2 & \ldots & \u_2^T \u_p \\[.4em]
  \vdots      & \vdots      & \ddots & \vdots \\[.4em]
  \u_p^T \u_1 & \u_p^T \u_2 & \ldots & \u_p^T \u_p \end{bmatrix}
  \pause
  = \begin{bmatrix}
    1 & 0 & \ldots & 0 \\ 
    0 & 1 & \ldots & 0 \\ 
    \vdots & \vdots & \ddots & \vdots \\
    0 & 0 & \ldots & 1 \end{bmatrix}_{p\times p} 
  \]
  }
  \vspace*{1em}
  
  \pause
  \begin{theorem}
  The columns of an $m \times n$ matrix $U$ form an \alert{orthonormal set} if and only if $U^TU=I_{n\times n}$.
  \end{theorem}
\end{frame}


\begin{frame}{Linear Mapping from Matrix whose Columns form an Orthonormal Set}
  \medskip

  \begin{theorem}
  Let $U$ be an $m \times n$ matrix with orthonormal columns, 
  and let $\x$ and $\y$ be in $\R^n$. Then
  \bb[(a)]
  \ii $(U \x) \cdot (U \y) = \x \cdot \y$.  \smallskip
  \ii $\| U\x \| = \| \x \|$.  \smallskip
  \ii $(U \x) \cdot (U \y) = 0$ if and only if $\x \cdot \y=0$.  \smallskip
  \ee
  In other words, the linear mapping $\x \mapsto U\x$ preserves \alert{length} and \alert{orthogonality}.
  \end{theorem}
  \bigskip

  \pause
  \blue{Proof.}
  
  Statements (b) and (c) follow from statement (a).
  Note that
    \[ (U \x) \cdot (U \y) = (U \x)^T (U \y) = \x^T (U^T U) \y = \x^T \y = \x \cdot \y. \]
    
  \vspace*{-2em}
  \hfill\blue{$\qed$}
\end{frame}

\end{document}
