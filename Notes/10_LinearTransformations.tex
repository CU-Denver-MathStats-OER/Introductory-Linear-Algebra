\documentclass[xcolor=dvipsnames,aspectratio=169,t]{beamer}
  % t means frames are vertically centered to the top
\usepackage{slides-header}
\title{Linear Transformations}

\begin{document}
\maketitle

\begin{frame}{Interpretations of the Matrix Equation $A \mathbf{x}=\mathbf{b}$}

  \bi
  \ii Our focus has been on using matrices and vectors to solve systems of linear equations.
  \ii We have applied these techniques to determine whether a vector is in the span of a set of vectors.
  \ii Now we will interpret the matrix-vector product $A \mathbf{x}$ as a \alert{function mapping from one space into another.}
  \ei
  \bigskip
  
  \pause
  \begin{example}

    \begin{columns}[T]
      \column{0.475\tw}
    \bi
    \ii $f: x \mapsto x^2$ maps from $\mathbb{R} \to \mathbb{R}$. \medskip
    \ii $f: x \mapsto 7x$   maps from $\mathbb{R} \to \mathbb{R}$.\medskip
     \ii $\dsty f: \begin{bmatrix} x_1 \\ x_2 \end{bmatrix} \mapsto \begin{bmatrix} 7x_1\\  7x_2 \end{bmatrix}$ maps from $\mathbb{R}^2 \to \mathbb{R}^2$.
    \ei

    \column{0.525\tw}
    \bi
%    \ii $\dsty f: \begin{bmatrix} x_1 \\ x_2 \end{bmatrix} \mapsto x_1 + x_2$ maps from $\mathbb{R}^2 \to \mathbb{R}$.\ms
    \ii $\dsty f: \begin{bmatrix} x_1 \\ x_2 \end{bmatrix} \mapsto \begin{bmatrix} x_2 \\ x_1 \\ x_1 + x_2 \end{bmatrix}$ maps from $\mathbb{R}^2 \to \mathbb{R}^3$. \ms
    %\ii In general: $\dsty T: \mathbf{x} \mapsto A\mathbf{x}$
    \ei
    \end{columns}
  \end{example}

  \end{frame}

\begin{frame}{Transformations from $\mathbb{R}^n$ to $\mathbb{R}^m$}

\bbox
A \alert{transformation} (also called a function or mapping) $T$ from $\mathbb{R}^n$ to $\mathbb{R}^m$ is a rule that assigns each vector $\mathbf{x}$ in $\mathbb{R}^n$ to a vector $T(\mathbf{x})$ in $\mathbb{R}^m$.
\bi
\ii The set of inputs, $\mathbb{R}^n$, is called the \alert{domain} of $T$.
\ii The space where the inputs are mapped to, $\mathbb{R}^m$, is call the \colorb{codomain} of $T$.
\ii For each $\mathbf{x}$ in the domain, the vector $T(\mathbf{x})$ in $\mathbb{R}^m$ is called the \colorg{image} of $\mathbf{x}$.
\ii The set of all images (outputs) is called the \colorg{range} of $T$.
\ei
\ebox
\medskip

\pause
\hspace*{18em} 
$\dsty T\left(\begin{bmatrix} x_1 \\ x_2 \end{bmatrix}\right) = 
  \begin{bmatrix}x_1+x_2 \\ x_2 \\ 0 \end{bmatrix}$

\end{frame}

\begin{frame}{Matrix Transformations}

  Let $A$ be an $m\times n$ matrix.  Then we can define a transformation $T\colon\R^n\to\R^m$ where \[ T\colon \x \mapsto A\x. \]
  \medskip
  
  \pause
  \bb
    \item $A=\begin{bmatrix} 7 & 0\\0&7 \end{bmatrix}$.
      \quad $T( \mathbf{x}) = \begin{bmatrix} 7 & 0\\0&7 \end{bmatrix} \begin{bmatrix} x_1 \\ x_2 \end{bmatrix} = \begin{bmatrix} 7x_1\\  7x_2 \end{bmatrix} \text{maps from $\R^2 \to \R^2$.}$

  \vspace*{2em}
  
  \pause
  \item $\dsty T\colon \begin{bmatrix} x_1 \\ x_2 \end{bmatrix} \mapsto \begin{bmatrix} x_2 \\ x_1 \\ x_1 + x_2 \end{bmatrix}$ maps from $\mathbb{R}^2 \to \mathbb{R}^3$. \

  \[ T( \mathbf{x}) = \begin{bmatrix} 0 & 1\\1&0\\1&1 \end{bmatrix} \begin{bmatrix} x_1 \\ x_2 \end{bmatrix}
  = \begin{bmatrix} x_2 \\ x_1 \\ x_1 + x_2 \end{bmatrix}\]

  \ee
\end{frame}


\begin{frame}{Example}
\medskip

  \begin{columns}
    \column{0.3\tw}
  Let $A=\dsty \begin{bmatrix} 1 & -3 & 2\\
    -4 & 1 & 0\\
    5 & 3 & -9 \end{bmatrix}$,
    \medskip
    
    and let $T$ be the matrix transformation $T\colon \x \mapsto A\x$.
    
  \column{0.7\tw}

  {\small
  \bb
  \vspace*{-2.5em}
  \ii Find the image of the vectors $\dsty \mathbf{e_1} = \begin{bmatrix} 1\\0\\0 \end{bmatrix}$, $\dsty \mathbf{e_2} = \begin{bmatrix} 0\\1\\0 \end{bmatrix}$, and $\dsty \mathbf{e_3} = \begin{bmatrix} 0\\0\\1 \end{bmatrix}$.%
  \vspace*{6em}
  
  \pause
  \ii Is $\dsty \mathbf{b} = \begin{bmatrix}1\\1\\1 \end{bmatrix}$ in the range of $T$? If so, what is the preimage?
  \ee}

  \end{columns}
\vspace{3in}

\end{frame}

\begin{frame}{Linear Transformations}

  \bbox
 A transformation $T: \mathbb{R}^n \to \mathbb{R}^m$ is called a \alert{linear transformation} if it satisfies:
  \bb
  \ii $T( \u + \v ) = T(\u)+ T(\v)$ \alert{for all} vectors $\u$ and $\v$ in $\R^n$.
  \ii $T(c \v) = c \big( T(\v) \big)$ \alert{for all} scalars $c$ and vectors $\v$ in $\R^n$.
  \ee
  \ebox
  \bigskip
  
  \pause
  $T\left(\begin{bmatrix} x_1 \\ x_2 \end{bmatrix}\right)
  =\begin{bmatrix}x_1+x_2 \\ x_2 \\ 0 \end{bmatrix}$ is a linear transformation.

\end{frame}

\begin{frame}{Matrix Transformations are Linear Transformations}
  \begin{theorem}
    Every \colorb{matrix transformation} is a \alert{linear transformation}. Namely, if $T\colon\mathbb{R}^n \to \mathbb{R}^m$ can be expressed as $T(\mathbf{x}) = A\mathbf{x}$ for some matrix $A$, then $T$ is a linear transformation.
    \end{theorem}

  \colorb{Proof.}
  
  
  \end{frame}

\begin{frame}{Image of the Zero Vector}

  \begin{theorem}
    If $T\colon\mathbb{R}^n \to \mathbb{R}^m$ is a linear map, then $T(\mathbf{0}) = \mathbf{0}$.
  \end{theorem}

  \colorb{Proof.}

\end{frame}


\begin{frame}{Example}

  Consider the linear transformation $T: \mathbb{R}^4 \to \mathbb{R}^4: \mathbf{x} \mapsto A\mathbf{x}$ where $\dsty A = \begin{bmatrix}
    4 & -2 & 5 & -5\\
    -9 & 7 & -8 & 0\\
    -6 & 4 & 5 & 3\\
    5 & -3 & 8 & -4 \end{bmatrix}$.

  \bb
  \item Find all $\mathbf{x}$ such that $T(\mathbf{x}) = \mathbf{0}$.
  \vspace*{6em}
  
  \item Find all $\mathbf{x}$ (if any) such that
    $\dsty T(\mathbf{x}) = \begin{bmatrix} 7\\ 5\\ 9\\ 7 \end{bmatrix}$.
  \ee
\end{frame}
  
\end{document}
