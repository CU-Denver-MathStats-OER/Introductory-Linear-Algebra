\documentclass[xcolor=dvipsnames,aspectratio=169,t]{beamer}
  % t means frames are vertically centered to the top
\usepackage{slides-header}
\title{The Invertible Matrix Theorem}

\begin{document}
\maketitle

\begin{frame}{Equivalent Statements}

Let $A$ be a square $n \times n$ matrix. 

\bb[(a)]
\ii $A$ is an invertible matrix.
\ii $A$ is row equivalent to the $n \times n$ identity matrix $I_n$.
\ii $A$ has $n$ pivots when forming the reduced row echelon form of $A$.
\ii The equation $A \mathbf{x} = \mathbf{0}$ has only the trivial solution.
%\addtocounter{enumi}{5}
%\ii There is an $n \times n$ matrix $C$ such that $CA = I_n$.
\ee
  \bigskip

  \pause
  \begingroup\addtolength{\leftmargini}{2em}
  \begin{itemize}
  \item[(d)$\Rightarrow$(b)] If $A \mathbf{x} = \mathbf{0}$ has only the trivial solution, that means the augmented matrix $\lbrack A \mathbf{0} \rbrack$ has no free variables. Thus $A$ has $n$ pivots. \ss
  \item[(c)$\Leftrightarrow$(b)] $A$ has $n$ pivots if and only if $A$ is row equivalent to the $n \times n$ identity matrix. \ss
  \pause
  \item[(a)$\Rightarrow$(d)] If $A$ is invertible, then $A\x=\b$ has a unique solution for any $b$ in $\mathbb{R}^n$.
  \item[(b)$\Rightarrow$(a)]
      If $A$ is row equivalent to $I_n$, then there are elementary matrices $E_i$ such that $E_p E_{p-1} \cdots E_1 A = I_n$.  Thus, $A^{-1} = E_p E_{p-1} \cdots E_1$, so $A$ is invertible.
  \end{itemize}
  \endgroup

  \pause
  \alert{Statements (a), (b), (c), (d) are all equivalent!}
\end{frame}

\begin{frame}{Equivalence of Statements}

  \bb[(a)]
  \ii $A$ is an invertible matrix.
  \ii $A$ is row equivalent to the $n \times n$ identity matrix, $I_n$.
  \ii $A$ had $n$ pivots.
  \ii The equation $A \mathbf{x} = \mathbf{0}$ has only the trivial solution.
  \addtocounter{enumi}{5}
  \ii There is an $n \times n$ matrix $C$ such that $CA = I_n$.
  \ee
  \bigskip

  \pause
  \begingroup\addtolength{\leftmargini}{2em}
  \begin{itemize}
    \item[(a)$\Rightarrow$(j)] If $A$ is invertible, then $C=A^{-1}$ is a matrix such that $CA = I_n$. \ss
    \item[(j)$\Rightarrow$(d)] If there is an $n \times n$ matrix $C$ such that $CA = I_n$, then $A \mathbf{x} = \mathbf{0}$ has only the trivial solution.
  \end{itemize}
  \endgroup
  \bigskip
  
  \pause
  \alert{Statements (a), (b), (c), (d), and (j) are all equivalent!}

\end{frame}

\begin{frame}{Adding to the Chain of Equivalence}

 \bb[(a)]
  \ii $A$ is an invertible matrix.
  \addtocounter{enumi}{5}
  \ii The equation $A \mathbf{x} = \mathbf{b}$ has at least one solution for each $\mathbf{b}$ in $\mathbb{R}^n$.
  \addtocounter{enumi}{3}
  \ii There is an $n \times n$ matrix $D$ such that $AD = I_n$.
  \ee
  \bigskip

  \pause
  \begingroup\addtolength{\leftmargini}{2em}
  \begin{itemize}
    \item[(a)$\Rightarrow$(k)] If $A$ is invertible, then there is an $n \times n$ matrix $D=A^{-1}$ such that $AD = I_n$.
    \item[(k)$\Rightarrow$(g)] If there is an $n \times n$ matrix $D$ such that $AD = I_n$, then the equation $AD \mathbf{b} = I_n \mathbf{b}=\mathbf{b}$. Thus, for any $\mathbf{b}$ in $\mathbb{R}^n$, the equation $A \mathbf{x} = \mathbf{b}$ has the solution $\mathbf{x} = D \mathbf{b}$.
    \item[(g)$\Rightarrow$(a)] If the equation $A \mathbf{x} = \mathbf{b}$ has at least one solution for each $\mathbf{b}$ in $\mathbb{R}^n$, each row in $A$ must have a pivot.
    Since $A$ is $n\times n$, $A$ has $n$ pivots, and $A$ therefore is invertible.
  \end{itemize}
  \endgroup
  \bigskip

  \pause
  \alert{Statements (a), (b), (c), (d),  (g), (j)  and (k) are all equivalent statements!}
\end{frame}

% \begin{frame}{Summary of Equivalence So Far}
%   \bb[(a)]
%   \addtocounter{enumi}{6}
%   \ii The equation $A \mathbf{x} = \mathbf{b}$ has at least one solution for each $\mathbf{b}$ in $\mathbb{R}^n$.
%   \ii The columns of $A$ span $\mathbb{R}^n$
%   \ii The linear transformation $\mathbf{x} \mapsto A\mathbf{x}$ maps $\mathbb{R}^n$ onto $\mathbb{R}^n$.
%   \ee
% 
% \bs
%   
%   \bb[(a)]
%   \addtocounter{enumi}{3}
%   \ii The equation $A \mathbf{x} = \mathbf{0}$ has only the trivial solution.
%   \ii The columns of $A$ form a linearly independent set.
%   \ii The linear transformation $\mathbf{x} \mapsto A\mathbf{x}$ is one-to-one.
%   \ee
% 
%   \bbox
%   \alert{Statements (a), (b), (c), (d),  (e), (f), (g), (h), (i), (j)  and (k) are all equivalent and statements!}
% \ebox
%   
% \end{frame}

\begin{frame}{The Invertible Matrix Theorem (so far)}

Let $A$ be a square $n \times n$ matrix. Then all of the following statements are equivalent.
  
 \bb[(a)]
  \ii \colorb{$A$ is an invertible matrix.}
  \ii $A$ is row equivalent to the $n \times n$ identity matrix $I_n$.
  \ii $A$ has $n$ pivots.
  \ii The equation $A \mathbf{x} = \mathbf{0}$ has only the trivial solution.
  %\ii The columns of $A$ form a linearly independent set.
  %\ii The linear transformation $\mathbf{x} \mapsto A\mathbf{x}$ is one-to-one.
  \addtocounter{enumi}{2}
  \ii The equation $A \mathbf{x} = \mathbf{b}$ has at least one solution for each $\mathbf{b}$ in $\mathbb{R}^n$.
  \ii The columns of $A$ span $\mathbb{R}^n$.
  %\ii The linear transformation $\mathbf{x} \mapsto A\mathbf{x}$ maps $\mathbb{R}^n$ onto $\mathbb{R}^n$.
  \addtocounter{enumi}{1}
  \ii There is an $n \times n$ matrix $C$ such that $CA = I_n$.
  \ii There is an $n \times n$ matrix $D$ such that $AD = I_n$.
  \ii \colorb{$A^T$ is an invertible matrix.}
\ee

\end{frame}

\begin{frame}{Example}

  Consider the matrix
  \[ A = \begin{bmatrix} 2 & 3 & 4\\ 2& 3 & 4\\2 & 3 & 4\end{bmatrix} .\]

  Does $A$ have an inverse? Explain why or why not.
  \medskip
  
  \pause
  \bi
  %\ii The columns of $A$ are not linearly independent.
  \ii $A$ is not row equivalent to $I_3$.
  \ii The columns of $A$ do not span $\mathbb{R}^3$.
  \ii $A^T$ is not invertible.
  \ii And so on $\ldots$
  \ei

\end{frame}

\end{document}
